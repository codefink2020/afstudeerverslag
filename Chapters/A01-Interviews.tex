% Appendix A

\chapter{Interviews \& gesprekken}\label{app:Interviews}

%TODO: MarginPARS zetten
In deze appendix zijn de verslagen te vinden van de verschillende interviews die gehouden zijn door de auteur met de verschillende betrokkenen.

\section{Intake Gesprek CTO }\label{sec:intake-gesprek}
\subsubsection{Doel}

Het doel van dit gesprek is op een lijn brengen van de opdracht en mijn interpretatie ervan. Dit gesprek is bepalend voor het verdere verloop van de opdracht en de scope die het mee geeft.

\subsubsection{Opzet}
Het gesprek heeft een open structuur waarbij er een leidraad is in de vragen die ik heb opgesteld voorafgaand aan het gesprek

\subsubsection{Verslag}
\textbf{Inleiding}
Aangegeven wat het doel is van het gesprek: requirements gathering en het vaststellen van stakeholders die in latere gesprekken geinterviewt kunnen worden over hun requirements.
Afgesproken is ook dat er gesproken wordt in je en jij.

\bigskip

\textbf{Hoe wordt op dit moment de SOUP-analyse gedaan en hoeveel tijd}
    Wat is de huidig situatie volgens jou, Hoe wordt er op dit moment een zorg gedragen dat de software die er gebouwd wordt veilig is voor productie?}

\lipsum[01]
\bigskip

\textbf{Welke methodes en principes worden er op dit moment toegepast om onveilige software uit de development stack te houden.}

\lipsum[03]
\bigskip

\textbf{Vraag4: Nu je terug kijkt op de opdracht die gegeven is, zijn er toevoegingen die nu, 4 weken na het uitbrengen van de opdracht, bestaan? Of zijn er zaken veranderd ten opzicht van inzichten die in de tussentijd zijn ontstaan.}

\lipsum[05]
\bigskip

\textbf{Vraag5: Welke Stakeholders zie jij voor dit project, wie heeft er het meeste nut van de nieuwe module? }

\lipsum[06]
\bigskip

\textbf{Vraag6: Wie zijn er op het moment bezig met de ontwikkeling van portal en kan ik inschakkelen als ik hulp nodig heb tijdens de implementatie?}

\lipsum[09]
\bigskip

\textbf{Vraag4: }

\lipsum[07]

\subsubsection{Resultaat?}

\subsection{Interview met projectmanager als stakeholder van de nieuwe module}

\subsubsection{Doel}
Het doel van dit gesprek is het verkrijgen van duidelijkheid over requirements en de aanwijzing van andere stakeholders voor de module.

\subsubsection{Opzet}
Het gesprek heeft een open structuur waarbij er een leidraad is in de vragen die ik heb opgesteld voorafgaand aan het gesprek

\subsubsection{Verslag}
\textbf{Inleiding}
Aangegeven wat het doel is van het gesprek: requirements gathering en het vaststellen van stakeholders die in latere gesprekken geinterviewt kunnen worden over hun requirements. Afgesproken is ook dat er gesproken wordt in je en jij.

\bigskip

\textbf{Vraag1: Wat is de huidig situatie volgens jou, Hoe wordt er op dit moment een zorg gedragen dat de software die er gebouwd wordt veilig is voor productie?}

\lipsum[01]
\bigskip

\textbf{Vraag2: In de opdracht staat vermeld welke eisen er gesteld staan aan de module, hoe zie je de werkwijze in de toekomst ten opzicht van nu?}

\lipsum[03]
\bigskip

\textbf{Vraag4: Nu je terug kijkt op de opdracht die gegeven is, zijn er toevoegingen die nu, 4 weken na het uitbrengen van de opdracht, bestaan? Of zijn er zaken veranderd ten opzicht van inzichten die in de tussentijd zijn ontstaan.}

\lipsum[05]
\bigskip

\textbf{Vraag5: Welke Stakeholders zie jij voor dit project, wie heeft er het meeste nut van de nieuwe module? }

\lipsum[06]
\bigskip

\textbf{Vraag6: Wie zijn er op het moment bezig met de ontwikkeling van portal en kan ik inschakkelen als ik hulp nodig heb tijdens de implementatie?}

\lipsum[09]
\bigskip

\textbf{Vraag4: }

\lipsum[07]

\subsubsection{Resultaat?}
\subsection{Interview met (senior) developer als stakeholder van de nieuwe module.}

\subsubsection{Doel}
Het doel van dit gesprek is het verkrijgen van duidelijkheid over requirements en de aanwijzing van andere stakeholders voor de module.

\subsubsection{Opzet}
Het gesprek heeft een open structuur waarbij er een leidraad is in de vragen die ik heb opgesteld voorafgaand aan het gesprek

\subsubsection{Verslag}
\textbf{Inleiding}
Aangegeven wat het doel is van het gesprek: requirements gathering en het vaststellen van stakeholders die in latere gesprekken geinterviewt kunnen worden over hun requirements. Afgesproken is ook dat er gesproken wordt in je en jij.

\bigskip

\textbf{Vraag1: Wat is de huidig situatie volgens jou, Hoe wordt er op dit moment een zorg gedragen dat de software die er gebouwd wordt veilig is voor productie?}

\lipsum[01]
\bigskip

\textbf{Vraag2: In de opdracht staat vermeld welke eisen er gesteld staan aan de module, hoe zie je de werkwijze in de toekomst ten opzicht van nu?}

\lipsum[03]
\bigskip

\textbf{Vraag4: Nu je terug kijkt op de opdracht die gegeven is, zijn er toevoegingen die nu, 4 weken na het uitbrengen van de opdracht, bestaan? Of zijn er zaken veranderd ten opzicht van inzichten die in de tussentijd zijn ontstaan.}

\lipsum[05]
\bigskip

\textbf{Vraag5: Welke Stakeholders zie jij voor dit project, wie heeft er het meeste nut van de nieuwe module? }

\lipsum[06]
\bigskip

\textbf{Vraag6: Wie zijn er op het moment bezig met de ontwikkeling van portal en kan ik inschakkelen als ik hulp nodig heb tijdens de implementatie?}

\lipsum[09]
\bigskip

\textbf{Vraag4: }

\lipsum[07]

\subsubsection{Resultaat?}


\section{Onderzoek architectuur Eaglescience}
\subsection{Interview Senior Developer t.b.v dev-stack onderzoek}

\subsubsection{Doel}
Het doel van dit gesprek is het verkrijgen van duidelijkheid over requirements en de aanwijzing van andere stakeholders voor de module.

\subsubsection{Opzet}
Het gesprek heeft een open structuur waarbij er een leidraad is in de vragen die ik heb opgesteld voorafgaand aan het gesprek

\subsubsection{Verslag}
\textbf{Inleiding}
Aangegeven wat het doel is van het gesprek: requirements gathering en het vaststellen van stakeholders die in latere gesprekken geinterviewt kunnen worden over hun requirements. Afgesproken is ook dat er gesproken wordt in je en jij.

\bigskip

\textbf{Vraag1: Wat is de huidig situatie volgens jou, Hoe wordt er op dit moment een zorg gedragen dat de software die er gebouwd wordt veilig is voor productie?}

\lipsum[01]
\bigskip

\textbf{Vraag2: In de opdracht staat vermeld welke eisen er gesteld staan aan de module, hoe zie je de werkwijze in de toekomst ten opzicht van nu?}

\lipsum[03]
\bigskip

\textbf{Vraag4: Nu je terug kijkt op de opdracht die gegeven is, zijn er toevoegingen die nu, 4 weken na het uitbrengen van de opdracht, bestaan? Of zijn er zaken veranderd ten opzicht van inzichten die in de tussentijd zijn ontstaan.}

\lipsum[05]
\bigskip

\textbf{Vraag5: Welke Stakeholders zie jij voor dit project, wie heeft er het meeste nut van de nieuwe module? }

\lipsum[06]
\bigskip

\textbf{Vraag6: Wie zijn er op het moment bezig met de ontwikkeling van portal en kan ik inschakkelen als ik hulp nodig heb tijdens de implementatie?}

\lipsum[09]
\bigskip

\textbf{Vraag4: }

\lipsum[07]

\subsubsection{Resultaat?}

\subsection{Interview Project manager t.b.v tooling}

\subsubsection{Doel}
Het doel van dit gesprek is het verkrijgen van duidelijkheid over requirements en de aanwijzing van andere stakeholders voor de module.

\subsubsection{Opzet}
Het gesprek heeft een open structuur waarbij er een leidraad is in de vragen die ik heb opgesteld voorafgaand aan het gesprek

\subsubsection{Verslag}
\textbf{Inleiding}
Aangegeven wat het doel is van het gesprek: requirements gathering en het vaststellen van stakeholders die in latere gesprekken geinterviewt kunnen worden over hun requirements. Afgesproken is ook dat er gesproken wordt in je en jij.

\bigskip

\textbf{Vraag1: Wat is de huidig situatie volgens jou, Hoe wordt er op dit moment een zorg gedragen dat de software die er gebouwd wordt veilig is voor productie?}

\lipsum[01]
\bigskip

\textbf{Vraag2: In de opdracht staat vermeld welke eisen er gesteld staan aan de module, hoe zie je de werkwijze in de toekomst ten opzicht van nu?}

\lipsum[03]
\bigskip

\textbf{Vraag4: Nu je terug kijkt op de opdracht die gegeven is, zijn er toevoegingen die nu, 4 weken na het uitbrengen van de opdracht, bestaan? Of zijn er zaken veranderd ten opzicht van inzichten die in de tussentijd zijn ontstaan.}

\lipsum[05]
\bigskip

\textbf{Vraag5: Welke Stakeholders zie jij voor dit project, wie heeft er het meeste nut van de nieuwe module? }

\lipsum[06]
\bigskip

\textbf{Vraag6: Wie zijn er op het moment bezig met de ontwikkeling van portal en kan ik inschakkelen als ik hulp nodig heb tijdens de implementatie?}

\lipsum[09]
\bigskip

\textbf{Vraag4: }

\lipsum[07]

\subsubsection{Resultaat?}

\section{Onderzoek architectuur SOUP analyse}
\subsection{Interview Senior Developer t.b.v SOUP analyse}

\subsubsection{Doel}
Het doel van dit gesprek is het verkrijgen van duidelijkheid over requirements en de aanwijzing van andere stakeholders voor de module.

\subsubsection{Opzet}
Het gesprek heeft een open structuur waarbij er een leidraad is in de vragen die ik heb opgesteld voorafgaand aan het gesprek

\subsubsection{Verslag}
\textbf{Inleiding}
Aangegeven wat het doel is van het gesprek: requirements gathering en het vaststellen van stakeholders die in latere gesprekken geinterviewt kunnen worden over hun requirements. Afgesproken is ook dat er gesproken wordt in je en jij.

\bigskip

\textbf{Vraag1: Wat is de huidig situatie volgens jou, Hoe wordt er op dit moment een zorg gedragen dat de software die er gebouwd wordt veilig is voor productie?}

\lipsum[01]
\bigskip

\textbf{Vraag2: In de opdracht staat vermeld welke eisen er gesteld staan aan de module, hoe zie je de werkwijze in de toekomst ten opzicht van nu?}

\lipsum[03]
\bigskip

\textbf{Vraag4: Nu je terug kijkt op de opdracht die gegeven is, zijn er toevoegingen die nu, 4 weken na het uitbrengen van de opdracht, bestaan? Of zijn er zaken veranderd ten opzicht van inzichten die in de tussentijd zijn ontstaan.}

\lipsum[05]
\bigskip

\textbf{Vraag5: Welke Stakeholders zie jij voor dit project, wie heeft er het meeste nut van de nieuwe module? }

\lipsum[06]
\bigskip

\textbf{Vraag6: Wie zijn er op het moment bezig met de ontwikkeling van portal en kan ik inschakkelen als ik hulp nodig heb tijdens de implementatie?}

\lipsum[09]
\bigskip

\textbf{Vraag4: }

\lipsum[07]

\subsubsection{Resultaat?}

\subsection{Interview Project manager informatie voorziening }

\subsubsection{Doel}
Het doel van dit gesprek is het verkrijgen van duidelijkheid over requirements en de aanwijzing van andere stakeholders voor de module.

\subsubsection{Opzet}
Het gesprek heeft een open structuur waarbij er een leidraad is in de vragen die ik heb opgesteld voorafgaand aan het gesprek

\subsubsection{Verslag}
\textbf{Inleiding}
Aangegeven wat het doel is van het gesprek: requirements gathering en het vaststellen van stakeholders die in latere gesprekken geinterviewt kunnen worden over hun requirements. Afgesproken is ook dat er gesproken wordt in je en jij.

\bigskip

\textbf{Vraag1: Wat is de huidig situatie volgens jou, Hoe wordt er op dit moment een zorg gedragen dat de software die er gebouwd wordt veilig is voor productie?}

\lipsum[01]
\bigskip

\textbf{Vraag2: In de opdracht staat vermeld welke eisen er gesteld staan aan de module, hoe zie je de werkwijze in de toekomst ten opzicht van nu?}

\lipsum[03]
\bigskip

\textbf{Vraag4: Nu je terug kijkt op de opdracht die gegeven is, zijn er toevoegingen die nu, 4 weken na het uitbrengen van de opdracht, bestaan? Of zijn er zaken veranderd ten opzicht van inzichten die in de tussentijd zijn ontstaan.}

\lipsum[05]
\bigskip

\textbf{Vraag5: Welke Stakeholders zie jij voor dit project, wie heeft er het meeste nut van de nieuwe module? }

\lipsum[06]
\bigskip

\textbf{Vraag6: Wie zijn er op het moment bezig met de ontwikkeling van portal en kan ik inschakkelen als ik hulp nodig heb tijdens de implementatie?}

\lipsum[09]
\bigskip

\textbf{Vraag4: }

\lipsum[07]

\subsubsection{Resultaat?}

\subsection{Interview senior developer t.b.v tooling met SOUP analyse specifiek}

\subsubsection{Doel}
Het doel van dit gesprek is het verkrijgen van duidelijkheid over requirements en de aanwijzing van andere stakeholders voor de module.

\subsubsection{Opzet}
Het gesprek heeft een open structuur waarbij er een leidraad is in de vragen die ik heb opgesteld voorafgaand aan het gesprek

\subsubsection{Verslag}
\textbf{Inleiding}
Aangegeven wat het doel is van het gesprek: requirements gathering en het vaststellen van stakeholders die in latere gesprekken geinterviewt kunnen worden over hun requirements. Afgesproken is ook dat er gesproken wordt in je en jij.

\bigskip

\textbf{Vraag1: Wat is de huidig situatie volgens jou, Hoe wordt er op dit moment een zorg gedragen dat de software die er gebouwd wordt veilig is voor productie?}

\lipsum[01]
\bigskip

\textbf{Vraag2: In de opdracht staat vermeld welke eisen er gesteld staan aan de module, hoe zie je de werkwijze in de toekomst ten opzicht van nu?}

\lipsum[03]
\bigskip

\textbf{Vraag4: Nu je terug kijkt op de opdracht die gegeven is, zijn er toevoegingen die nu, 4 weken na het uitbrengen van de opdracht, bestaan? Of zijn er zaken veranderd ten opzicht van inzichten die in de tussentijd zijn ontstaan.}

\lipsum[05]
\bigskip

\textbf{Vraag5: Welke Stakeholders zie jij voor dit project, wie heeft er het meeste nut van de nieuwe module? }

\lipsum[06]
\bigskip

\textbf{Vraag6: Wie zijn er op het moment bezig met de ontwikkeling van portal en kan ik inschakkelen als ik hulp nodig heb tijdens de implementatie?}

\lipsum[09]
\bigskip

\textbf{Vraag4: }

\lipsum[07]

\subsubsection{Resultaat?}
