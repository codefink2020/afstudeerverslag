% Appendix A

\chapter{Interviews \& gesprekken}\label{app:Interviews}

%TODO: MarginPARS zetten
In deze appendix zijn de verslagen te vinden van de verschillende interviews die gehouden zijn door de auteur met de verschillende betrokkenen.
\section{Intake gesprek Oprachtgever}\label{sec:intake-gesprek-oprachtgever}

\subsection{Doel}\label{subsec:intakeDoel}
\lipsum[01]
\subsection{Vragen}\label{subsec:intakeVragen}
\lipsum[01]

\lipsum[01]

\lipsum[01]

\section{Interviews met collega's over de dev-stack die gebruikt wordt binnen EagleScience}\label{sec:dev-stackInterviews}


Het \textbf{doel} van deze gesprekken is inzicht krijgen in de reden waarom we bepaalde tools en ontwikkeltalen gebruiken binnen EagleScience. Bij alle gesprekken die gevoerd zijn is er aangegeven dat het verslag nagekeken mag worden. Er geanonimiseerd mag worden. Ook heb ik aangegeven waar nodig dat ik het gesprek opneem en alleen gebruik als input voor het onderzoek en de implementatie. Daarnaast heb ik er voor gekozen om semigestructureerde gesprekken te voeren waarin vooraf een doel heb verkondigd. De reden voor de beslissing is dat door de manier waarop EagleScience voor haar klanten werkt er niet veel tijd overblijft voor niet inplanbare uren.

\subsection{Gesprek over Scala met Bas Broere}\label{subsec:gesprek-over-scala-met-bas-broere}
Het \textbf{doel} van dit gesprek is om er achter te komen hoe en waarom we Scala gebruiken. Een aantal
\textbf{Introductie: }\\
Inleiding gegeven over het doel van het gesprek en dat het verslag nakeken mag worden en dat desgewenst delen kunnen worden geanonimiseert
\subsection{}\label{subsec:dev-stackVragen}

