% Chapter 1
\chapter{Eaglescience} % Chapter title

\label{ch:Eaglescience} % For referencing the chapter elsewhere, use \autoref{ch:introduction}

%----------------------------------------------------------------------------------------

Het hier beschreven onderzoek en de daarbij behorende applicatie is geschreven in opdracht van het bedrijf Eaglescience wat gevestigd is in Amsterdam Sloterdijk. Eaglescience ontwikkeld complexe software op projectbasis voor diverse klanten. Deze projecten hebben vaak een wetenschappelijke inslag. Hiernaast biedt Eaglescience ook de mogelijkheid aan de klant om zorg te dragen voor de eventuele hosting van het opgeleverde product. Eaglescience kan hierdoor nog beter garanderen dat de geboden kwaliteit in de software gewaarborgd blijft tijdens de levensduur van de software.

\section{Organisatie}
\marginpar{Organisatie}

Eaglescience BV bestaat uit drie divisies: Innovations, Software en Solutions [figuurX]: %besxchrijving van de drie takken
Het bestaat op het moment van schrijven uit $\pm$ 20 medewerkers waarvan 75\% verantwoordelijk is voor de ontwikkeling van de geleverde software. De andere 25\% bekleed een support rol zoals project manager, finance manager, quality manager, automatisering etc. \\

\begin{figure}[bth]
\myfloatalign
\includegraphics[width=10cm]{gfx/organogram}
\caption{Organogram Eaglescience}
\label{fig:Eaglescience organogram}
\end{figure}
% vragen Wat zijn de hoofddoelen van iedere divisie?
% betere verwoording vragen:
De divisie Eaglescience Innovations zoekt naar nieuwe oplossingen op het gebied van software ontwikkeling, deze worden door de divisie  Eaglescience software ge\"implementeert. Eaglescience Solutions is een divisie die samen met de klant opzoek gaat naareen oplossing voor een gesteld probleem.\\
Het dagelijks bestuur is handen van:
\begin{itemize}
\item CEO/CFO - Marc Grootjen
\item CTO - Bas Breier
\item COO - Wender van Mansvelt \\
\end{itemize}
Onder het dagelijks bestuur valt Team Eaglescience wat bestaat uit projectmanagers en ontwikkelaars. Deze zijn onderverdeeld in diverse scrum teams die ieders verantwoordelijk zijn voor een project. De ontwikkelaars worden parallel ingezet op meerdere projecten om kennisdeling te bevorderen.

\section{missie}
\marginpar{missie}
De missie van Eaglescience is het bedienen van haar partners door een ontwerp, ontwikkeling en service te bieden op het gebied van op maat gemaakte IT oplossingen . Om dit te kunnen bewerkstelligen heeft Eaglescience goed opgeleide IT professionals in dienst die zichzelf continue ontwikkelen op de “cutting edge” van IT technologie. De hoofd competenties van de medewerkers zijn: innovatief, intelligent, klant georientee\"erd, flexibel en ambitieus.

\section{visie}
Eaglescience streeft er als innovatief IT bedrijf naar om software te ontwikkelen als een Business-to-Business dienst. Met onze technische vaardigheden bouwen we veilige en hoogwaardige software die bijdraagt aan een betere wereld. Omdat we agile werken, leveren we precies wat nodig is, niets meer en niets minder. Wij helpen onze klanten zoeken naar een langdurige betrokkenheid en samenwerking op basis van zowel vertrouwen als wederzijds respect. \\
\marginpar{visie}
Omdat elke vraag uniek is, ontwikkeld Eaglescience op maar gemaakte en innovatieve software. We zijn van plan deel uit te maken van het hele proces van het formuleren van een idee tot het lanceren van het product en het waarborgen van de productie levenscyclus. Onze belangrijkste succesfactor zijn de mensen, die zich continu ontwikkelen door met de nieuwste technieken te werken op diverse projecten. Wij streven naar een optimale balans tussen werk en priv\'e. Dit geeft onze medewerkers veel vrijheid, maar vereist zelfdiscipline en verantwoordelijkheid.

\section{strategie}
Eaglescience levert de visie via vier strategische thema's:
\begin{itemize}
\item Maatschappelijke verantwoordelijkheid
\item Persoonlijke groei
\item Tevredenheid
\item 4e???? %hbjvjhbjb
\end{itemize}
\marginpar{strategische thema's}
We streven ernaar om veilige en hoogwaardige software diensten te leveren die waarde toevoegen aan onze samenleving. We streven naar een bedrijfscultuur waarin alle collega's hun talenten kunnen laten groeien. We hebben een ongecompliceerd werkethos: we richten ons op resultaten van hoge kwaliteit, maar met een gezonde balans tussen werk en priv\'e en voldoende tijd voor leuke en sociale evenementen. Eaglescience verwacht van alle medewerkers dat zij hun handelen baseren op vier kwaliteitsprincipes:
\begin{itemize}
\item Meld situaties die niet voldoen aan onze interne procedures
\item Evalueer risico's wanneer grote veranderingen worden verwacht
\item Help en daag elkaar uit
\item Kennis behouden over compliancy en kwaliteitsmanagement
\end{itemize}

\section{Werkwijze}
Zoals eerder gemeld werkt Eaglescience op projectbasis met ontwikkelaars in meedere teams. Er wordt getracht "Full scrum "te werken waarbij de requirements van de klant centraal staan. Als een project wordt aangenomen door het management team dan wordt deze in sprints in samenwerking met de klant ontwikkeld. De klant wordt nauw betrokken bij het verloop van de ontwikkeling door middel van demo's aan het einde van iedere sprint, hier wordt gemeten hoe de applicatie zich gedraagt met betrekking tot de requirements van de klant. Dit is ook het moment dat er feedback gegeven wordt en waar waar nodig gestuurd kan worden in het verdere verloop. Op het moment dat er een applicatie klaar is wordt de software dan al niet overgedragen aan de klant of door gegeven aan support en hosting die verantwoordelijk zijn voor de daadwerkelijke hosting van de software.
\begin{figure}[bth]
\myfloatalign
\includegraphics[width=10cm]{gfx/ProcessFlow}
\caption{Project Process}
\label{fig:Project Process}
\end{figure}
Naast het ontwikkel process zijn er een aantal supporting processes die ervoor zorgdragen dat het bedrijf blijft draaien en er nieuwe mensen aangenomen worden, maar ook een deel automatisering die voor ondersteuning zorgt voor platformen waarop ontwikkeld en of gehosted wordt.\\
Eaglescience ontwikkeld op projectbasis en op die manier komen er ook inkomsten. Dus alle processen die draaien moeten ingezet kunnen worden op projecten van klanten. Als er een project voor in huis gebruik wordt ondernomen moet er een duidelijk beeld zijn of er op termijn winst mee te behalen is op monetair vlak dan al niet tijdswinst of ontwikkel gemak.

\section{}

\section{Relevante en actuele ontwikkelingen binnen Eaglescience}
Eaglescience is aan het groeien, zowel in het aantal projecten waar aan gewerkt wordt als het aantal medewerkers. Daarnaast worden de diensten die Eaglescience aanbied ook uitgebreid. Waarbij het hosten van de ontwikkelde applicaties steeds meer wordt aangeboden. Door deze inzet ligt de verantwoordelijk niet alleen bij het leveren van een veilige en hoogwaardige software maar het leveren van service waarbij de applicaties in een veilige omgeving worden aangeboden. Mede door de groei van het bedrijf maar zeker ook de diensten die aangeboden wordt is het zeer relevant om taken die geautomatiseerd kunnen worden te automatiseren.
