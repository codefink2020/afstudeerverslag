% Appendix X

\chapter{BegrippenLijst [WIP]}

%----------------------------------------------------------------------------------------

% Content begins here
\textbf{Time to Market}\\
De marktintroductietijd is de tijdsduur benodigd om een product te ontwerpen totdat het op de markt verschijnt. De benodigde tijd om een product op de markt te brengen is zeer belangrijk in industrieën waar de levensduur van een product kort is. Bij een korte productlevenscyclus is het belangrijk, om winst te kunnen maken, om als eerste met het product op de markt te verschijnen.\\

\textbf{MoSCoW-methode}
De MoSCoW-methode is een wijze van prioriteiten stellen in onder meer de software engineering. De eisen aan het resultaat van een project worden ermee ingedeeld. Het is een afkorting, waarvan de letters staan voor:\\
\textit{M} - must haves: deze eisen (requirements) moeten in het eindresultaat terugkomen, zonder deze eisen is het product niet bruikbaar;\\
\textit{S} - should haves: deze eisen zijn zeer gewenst, maar zonder is het product wel bruikbaar;\\
\textit{C} - could haves: deze eisen zullen alleen aan bod komen als er tijd genoeg is;\\
\textit{W} - won't haves: deze eisen zullen in dit project niet aan bod komen maar kunnen in de toekomst, bij een vervolgproject, interessant zijn.\\
De o's in de afkorting hebben geen betekenis

\textbf{dev-stack}
gebruikte technologi\"en door een bedrijf om software te ontwikkelen. Hieronder vallen de verschillende talen, frameworks de gebruikt worden om te ontwikkelen maar ook tooling dat ondersteund bij het ontwikkelen van de software.....
