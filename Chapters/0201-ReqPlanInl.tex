% Chapter 3

\chapter{Inleiding}\label{ch:ReqInl}
[NOTE: ] nog wel nodig: gezien planning naar vorige deel is vervangen.
Ondanks dat er gewerkt wordt middels een Agile/scrum methode is het gewenst om vooraf een idee te krijgen over de applicatie die gebouwd dient te worden. Om dit inzicht te verkrijgen is wordt er een requirements analyse uitgevoerd welke als resultaat een Requirements specification heeft. Op basis van dit document zal in latere delen een ontwerp worden gemaakt en vervolgens een implementatie.

Bij de requirements analyse zullen de stakeholders nauw worden betrokken om op deze manier
een goed beeld te krijgen bij de verwachtingen van de module. Tijdens deze analyse zal ook worden gekeken naar de haalbaarheid van de requirements en de prioriteit die er aan vast hangt.
