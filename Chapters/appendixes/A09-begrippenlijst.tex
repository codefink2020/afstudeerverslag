% Appendix X

\chapter{BegrippenLijst}\label{ch:begrippenlijst}

%TODO: MarginPARS zetten

%----------------------------------------------------------------------------------------

% Content begins here

Goede manier vinden van onderscheiden zonder gebruik van secties....

\section*{Nog op alphabetische volgorde zetten!!!!!!}\label{sec:nog-op-alphabetische-volgorde-zetten!!!!!!}


\section{SOUP:}\label{sec:soup:} Software Of Unkown Provinence/Pedigree. is ook behandeld in de sectie~\ref{sec:dependency-trees?} in het kort is dit een weergave van dependencies en hun dependencies deze kan worden weergeven worden in een boom structuur.
\smallskip

\section*{Dev-Stack:}\label{sec:dev-stack:} De ontwikkelomgeving die gebruikt wordt door een bedrijf. Dit is meestal een opsomming van gebruikte technologiën zoals Programeer talen, frameworks, en buildtools. Vaak worden ook andere tools genoemd zoald IDE's en andere Editors.
\smallskip

\section*{Code smell:}\label{sec:code-smell:} De ontwikkelomgeving die gebruikt wordt door een bedrijf. Dit is meestal een opsomming van gebruikte technologiën zoals Programeer talen, frameworks, en buildtools. Vaak worden ook andere tools genoemd zoald IDE's en andere Editors.
\smallskip

\section{Linting:}\label{sec:linting:} De ontwikkelomgeving die gebruikt wordt door een bedrijf. Dit is meestal een opsomming van gebruikte technologiën zoals Programeer talen, frameworks, en buildtools. Vaak worden ook andere tools genoemd zoald IDE's en andere Editors.
\smallskip




\section{Dependency tree:}\label{sec:dependency-tree:} is ook behandeld in de sectie~\ref{sec:dependency-trees?} in het kort is dit een weergave van dependencies en hun dependencies deze kan worden weergeven worden in een boom structuur.
\smallskip

\section{Pure functie:}\label{sec:pure-functie:}
Een Pure functie is een functie die alleen een output genereerd op basis van een input dus als de functie \( y = x+1\) is dan geeft de functie bij een input van 2 dus 3 terug.
Een pure functie heeft dus geen side-effects die iets anders doen dan een output genereren op basis van de input.
Een applicatie bouwen met alleen maar pure functies is niet mogelijk gezien er nooit een I/O plaats kan vinden.
Deze I/O wordt dan ook meestal door een schil geregeld als in de volgende listing is te zien:
Zie \autoref{lst:pf} hieronder voor een voorbeeld.

%float=b,language=Scala,frame=tb, << Settings die eerst voor caption stonden
\begin{lstlisting}[caption={Pure functie met IO},label=lst:pf]


def abs(n:Int):Int =
  if(n<0) -n
  else n

def formatabs(x:Int): String = {
  val msg = "The absolute value of %d is %d"
  msg.format(x,abs(x))
}

//Unit is het Scala equivalent van Void in Java.
def main(args: Array[String]):Unit = {
  println(formatabs(-42)
}
\end{lstlisting}

Zoals te zien is de abs functie en pure functie gezien deze een input(Int) verwacht en alleen een output(Int) terug geeft.
Ook de format Abs is een pure functie er gaan twee input variabelen in en er komt altijd een String als output uit.
De waarde van de string is altijd hetzelfde bij dezelfde inputs.
De main functie is geen pure functie dit omdat er geen input en geen output is gedefineerd.
Echter, ontstaat er wel een output gezien er iets op de console wordt geprint door de println(formatabs(-42)) functie.
\smallskip

\section{JVM:}\label{sec:jvm:}
De JVM ook wel Java Virtual Machine is de runtime omgeving voor java applicaties.
Het voordeel is dat een JVM de runtime abstraheert van de os waardoor de applicaties geschreven in Java of een Java afgeleide taal kan worden uitgevoerd op verschillende bestuuringssystemen.
Dit wordt gedaan door middel van een compilatie van (Java)Sourcecode naar bytecode wat door de JVM wordt gecompileerd door het JIT (Just in Time) principe.
Dit komt de portabiliteit ten goede omdat er maar een enkele keer code geschreven hoef te worden wat zowel op mac, windows als linux hetzelfde gedraagt.
De JVM bied ondersteuning voor verschillende talen naast Java namelijk: Scala, Groovy en Kotlin.
\smallskip

\section{Time-to-market:}\label{sec:time-to-market:}
De marktintroductietijd is de tijdsduur benodigd om een product te ontwerpen totdat het op de markt verschijnt.
De benodigde tijd om een product op de markt te brengen is zeer belangrijk in industrieën waar de levensduur van een product kort is.
Bij een korte productlevenscyclus is het belangrijk, om winst te kunnen maken, om als eerste met het product op de markt te verschijnen.
\smallskip

\section{MoSCoW-methode:}\label{sec:moscow-methode:}
De MoSCoW-methode is een manier van prioriteiten stellen in onder meer de software-engineering.
De eisen aan het resultaat van een project worden ermee ingedeeld.
Het is een afkorting, waarvan de letters staan voor:

\textit{M} - must haves: deze eisen (requirements) moeten in het eindresultaat terugkomen, zonder deze eisen is het product niet bruikbaar;

\textit{S} - should haves: deze eisen zijn zeer gewenst, maar zonder is het product wel bruikbaar;

\textit{C} - could haves: deze eisen zullen alleen aan bod komen als er tijd genoeg is;

\textit{W} - won't haves: deze eisen zullen in dit project niet aan bod komen, maar kunnen in de toekomst bij een vervolgproject, interessant zijn.
De o's in de afkorting hebben geen betekenis
\smallskip
