
\chapter{Requirements Analyse} % Chapter title

\label{inOnderzoek} % For referencing the chapter elsewhere, use \autoref{ch:InOnderzoek}

\section{huidige situatie}
In de huidige situatie wordt er een SOUP analyse gedaan door de ontwikkelaars op het moment dat een project ontwikkelt wordt. dit is veelal handmatig zoeken in online resources op bibliotheken die gebruikt worden. Dit neemt veel kostbare tijd in beslag die beter besteed kan worden om nieuwe features toe te voegen. Daarnaast worden de bevindingen die gedaan worden niet centraal opgeslagen zodat er een potentie is dat niet iedereen op de hoogte is van de actuele informatie.

\section{De Stakeholders}
De stakeholders kunnen opgedeelt worden in twee groepen(Tabel 1): externe en interne stakeholders.
\begin{table}
%\myfloatalign
\begin{tabularx}{\textwidth}{Xll} \toprule
\tableheadline{Groep}   & \tableheadline{Stakeholder}\\
\midrule
Extern                  & Klant                      \\
\midrule
Intern                  & Dagelijks Bestuur          \\
                        & Project managers           \\
                        & CTO                        \\
                        & Ontwikkelaars              \\
\bottomrule
\end{tabularx}
\caption[Verdeling stakeholders]{Verdeling stakeholders}
\label{tab:verdeling_StakeHolders}
\end{table}

Een deel van de stakeholders zijn geinterviewt om te achterhalen welke verbeteringen zij zien in de nieuwe module.

Verslagen van de interviews zijn te lezen in bijlage X Interviews


\subsection{Dagelijks bestuur (intern) }
De CTO ziet vooral winst in tijd als deze module
\subsection{Klant(extern)}
De klant is een passieve stackeholder gezien zei benifiet hebben van de ontwikkeling van deze module. Het belangrijkste resultaat voor de klant is software die veiliger is. of waar de kwetsbaarheden van bekend zijn. Waarop de klant kan beslissen om deze kwetsbaarheden aan te pakken na een advies vanuit Eaglescience of een derde parij.

\subsection{Projectmanagers (intern)}
De projectmanagers hebben baat bij de uitkomst van de analyse. Dit is al waardevol in de huidige situatie, en zal dus niet veranderen in de toekomstige situatie. Wat vooral belangrijk is
\subsection{Ontwikkelaar (intern)}
De Ontwikkele
\subsection{Stakeholder analyse}
\section{Requirements}
