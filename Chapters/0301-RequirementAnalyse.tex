
\chapter{Requirements Analyse}\label{ch:requirements-analyse}
Dit hoofdstuk beschrijft de methode en stappen die zijn ondernomen om de requirements voor de nieuwe module te analyseren. De analyse heeft als resultaat een document dat in Appendix \ref{ch:requirements-specificatie} te vinden is. Daarnaast zal hier een samenvating te vinden zijn van dit document die als input geldt voor de volgende hoofdstukken.

\section{Analyse stappen}\label{sec:analyse-stappen}
Om een goede analyse te kunnen doen zullen er een aantal stappen  moeten worden doorlopen.
\textbf{Stap 1: Probleem analyse}
In deze stap dient het probleem dat gesteld wordt in de opdracht te worden geanalyseerd. Door als eerste te kijken naar de huidige methode, die gehanteerd wordt om een SOUP analyse te doen, is er inzicht in hoe het probleem dat gesteld is op dit moment wordt opgelost. Er kan bijvoorbeeld gekeken worden naar de tijd die ontwikkelaars nodig hebben om een analyse te doen en vervolgens de resultaten te publiseren. Daarnaast is een onderzoek naar de betrokkenen gewenst om in te zien wie er baat heeft bij verbeteringen. Daarnaast kunnen betrokkenen ook gewensten en ongewenste input geven in een mogelijke oplossing.
\textbf{Stap 2: Onderzoek naar verbeteringen.}
Door interviews te houden met de verschillende betrokkenen wordt er inzicht verkregen in een mogelijke oplossing voor het gestelde probleem. Daarnaast door te kijken naar de resultaten die door de huidige manier wordt gegenereerd kan er een manier worden bedacht om deze resultaten te evenaren al dan niet te verbeteren.

\textbf{Stap 3: Opstellen van de user-requirments en stories.}
De bevindingen uit de analyse van de opdracht en het onderzoek beschreven in stap 2 geven input voor het opstellen van requirements en userstories waaraan de nieuwe methode moet voldoen.

\section{Samenvating Requirements specificatie.}



Problem Analysis

The most straightforward (and probably the most commonly used) requirements-analysis technique is problem analysis. Problem analysis means asking the users and managers to identify problems with the as-is system and to describe how to solve them in the to-be system. Most users have a very good idea of the changes they would like to see, and most are quite vocal about suggesting them. Most changes tend to solve problems rather than capitalize on opportunities, but the latter is possible as well. Improvements from problem analysis tend to be small and incremental (e.g., provide more space in which to type the customer’s address; provide a new report that currently does not exist).
This type of improvement often is very effective at improving a system’s efficiency or ease of use. However, it often provides only minor improvements in business value—the new system is better than the old, but it may be hard to identify significant monetary benefits from the new system.







Om een goede requirements analyse te doen moet er gekeken worden naar de huidige situatie en waar mogelijk verbeteringen zijn. Om deze verbeteringen te onderzoeken zijn er interviews gehouden met ber






Om inzicht te krijgen in de huidige en gewenste situatie zijn er interviews


Dit hoofdstuk beschrijft het probleem binnen het domein van EagleScience. Daarnaast wordt de huidige situatie beschreven waarna er een gewenste situatie wordt geschetst. De stakeholders worden defineert en in kaart gebracht Waarna hun requirements in kaart worden gebracht en middels de MoSCoW methode op prioriteit worden gezet. Het resultaat is een Requirements Specificatie die in zijn geheel is terug te vinden in Appendix~\ref{ch:requirements-specificatie}


\textbf{Must Have Moet nog onderverdeeld worden in MoSCoW}
\begin{itemize}
  \item Als \textit{gebruiker} wil ik dat de SOUP module in de portal te vinden is zodat alle tools die gebruikt worden binnen Eaglescience op een enkele plek te vinden zijn.
  \item Als \textit{gebruiker} wil ik een overzicht per project kunnen zien met daarin de gebruikte bibliotheken zodat ik inzage heb ik wat er gebruikt wordt voor ontwikkeling.
  \item Als \textit{gebruiker} wil ik een overzicht per project zien welke kwetsbaarheden er zich in bibliotheken bevinden, zodat ik actie kan ondernemen om de software nog veiliger te maken.
  \item Als \textit{gebruiker} wil ik in kunnen loggen met mijn LDAP? account zodat ik niet nog een keer een username/wachtwoord combinatie hoe te leren.
  \item Als \textit{gebruiker} wil ik een project kunnen toevoegen zodat ik ook van dat project de kwetsbaarheden in kan zien en deze software ook veilger wordt.
  \item Als \textit{Module} wil ik een update krijgen van de laatste build met specifiek de laatste kwetsbaarheden, zodat ik deze kan weergeven in de portal.
  \item Als \textit{module} wil ik
  \item Als \textit{gebruiker} wil ik dat periodiek automatisch een check analyse wordt uitgevoerd zodat ik er zelf niet naar om hoef te kijken.
  \item Als \textit{gebruiker} wil ik zelf een analyse kunnen starten voor een project zodat ik een up-to-date versie heb van de resultaten.
  \item Als \textit{Project manager} wil ik projecten kunnen toevoegen aan de module, zodat ook deze mee genomen worden in de automatische analyse.
  \item Als \textit{Project manager} wil ik ontwikkelaars kunnen toevoegen aan een project zodat deze ook inzicht krijgen in de huidige stand van zaken.
  \item Als \textit{Project manager} wil ik een notificatie( via mail/rocketchat) ontvangen als er een
\end{itemize}

\textbf{Should Have}
\begin{itemize}
  \item Moeten nog voorkomen uit de prioriteit analyse
\end{itemize}

\textbf{Could Have}
\begin{itemize}
\item
\end{itemize}

\textbf{Won't Have}
\begin{itemize}
  \item Moeten nog voorkomen uit de prioriteit analyse
\end{itemize}
De Won't haves staan hierbij genoemd als leidraad voor eventueel updates in de toekomst.
Als blijkt dat er tussen de won'ts toch low hanging fruit blijkt te hangen kunnen deze meegenomen worden in de sprints.
De requirements worden als epics in een JIRA omgeving gezet om vervolgens een planning te kunnen maken.

%\section{kennis vergaring}\label{sec:kennis-vergaring}
%De secties over de huidige en gewenste situaties zijn voornamlijk gebasseerd op eigen observaties en gesprekken die ik heb gehad met enkele ontwikkelaars en de CTO. Uit deze gesprekken kwamen ook een aantal stakeholders naar voren die ook in het project moeten worden meegenomen. Deze zijn geinterviewd om informatie te verkrijgen over de wensen die zij hebben voor de nieuwe module maar ook het belang die zij hebben in de module. Het geen resulteerd in een document dat te lezen is in~\ref{ch:requirements-specificatie}
