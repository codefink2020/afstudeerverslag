
\chapter{Requirements Analyse}\label{ch:requirements-analyse}
Dit hoofdstuk beschrijft de methode en stappen die zijn ondernomen om de requirements voor de nieuwe module te analyseren. De analyse heeft als resultaat een document dat in Appendix~\ref{ch:requirements-specificatie} te vinden is. Daarnaast zal hier een samenvating te vinden zijn van dit document die als input geldt voor de volgende hoofdstukken.

\section{Analyse stappen}\label{sec:analyse-stappen}
Om een goede analyse te kunnen doen zullen er een aantal stappen  moeten worden doorlopen.
\textbf{Stap 1: Probleem analyse}
In deze stap dient het probleem dat gesteld wordt in de opdracht te worden geanalyseerd. Door als eerste te kijken naar de huidige methode, die gehanteerd wordt om een SOUP analyse te doen, is er inzicht in hoe het probleem dat gesteld is op dit moment wordt opgelost. Er kan bijvoorbeeld gekeken worden naar de tijd die ontwikkelaars nodig hebben om een analyse te doen en vervolgens de resultaten te publiseren. Daarnaast is een onderzoek naar de betrokkenen gewenst om in te zien wie er baat heeft bij verbeteringen. Daarnaast kunnen betrokkenen ook gewensten en ongewenste input geven in een mogelijke oplossing.
\textbf{Stap 2: Onderzoek naar verbeteringen.}
Door interviews te houden met de verschillende betrokkenen wordt er inzicht verkregen in een mogelijke oplossing voor het gestelde probleem. Daarnaast door te kijken naar de resultaten die door de huidige manier wordt gegenereerd kan er een manier worden bedacht om deze resultaten te evenaren al dan niet te verbeteren.

\textbf{Stap 3: Opstellen van de user-requirments en stories.}
De bevindingen uit de analyse van de opdracht en het onderzoek beschreven in stap 2 geven input voor het opstellen van requirements en userstories waaraan de nieuwe methode moet voldoen.

\section{Samenvating Requirements specificatie.}\label{sec:samenvating-requirements-specificatie.}

