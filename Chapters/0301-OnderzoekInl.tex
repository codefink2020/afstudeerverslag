% Chapter 3

\chapter{Inleiding} % Chapter title

\label{inOnderzoek} % For referencing the chapter elsewhere, use \autoref{ch:InOnderzoek}
Op basis van de requirements analyse beschreven in het vorige deel zijn er een aantal vragen ontstaan die verder onderzoek benodigd behoeven. In dit deel worden de vragen geanalyseerd en beantwoord zodat er een duidelijkheid is in de materie en een goede basis wordt gelegd voor de daadwerkelijke implementatie beschreven in het volgende deel.

\section{Scope}
Het onderzoek zal zich beperken tot de benodigde informatie voor het implementeren van de nieuwe oplossing voor een geautomatiseerde SOUP analyse. Het zal ingaan op de gebruikte ontwikkelstack binnen Eaglescience en bestaande architectuur gezien de nieuwe oplossing een onderdeel is van een al bestaand project en hier dus naatloos op moet integreren. Daarnaast zal er onderzoek gedaan worden naar wat een SOUP analyse daadwerkelijk is en welke problemen het mogelijk op kan lossen. Met daarbij een mogelijke oplossingen om een SOUP analyse te kunnen doen.

Gezien de vragen in twee verschillende domeinen gesteld worden is het ook noodzakelijk om deze vragen op te delen in twee onderzoeken. In de komende hoofdstukken zal er dan ook voor ieder domein een eigen onderzoek worden beschreven met daarin de resultaten die vervolgens gebruikt kunnen worden voor de implementatie die volgt in een volgend deel.

De volgende onderwerpen worden in deze hoofdstukken beschreven:
\begin{itemize}
\item Onderzoeksmethode, een beschrijving van de gebruikte methoden en aanpak van de beide onderzoeken.
\item Onderzoek: Architectuur binnen eaglescience, een onderzoek naar de gebruikte architectuur binnen Eaglescience alsook de werkwijze waarop Eaglescience software ontwikkeld.
\item Onderzoek: SOUP-analyse, een onderzoek over wat SOUP precies is welke gevaren er potentieel mee gemoeid gaan, en welke oplossingen er bestaan om SOUP-analyses te doen.
\end{itemize}
