% Chapter 3

\chapter{Inleiding} % Chapter title

\label{inOnderzoek} % For referencing the chapter elsewhere, use \autoref{ch:InOnderzoek}
Op basis van de requirements analyse beschreven in het vorige deel zijn er een aantal vragen ontstaan die verder onderzoek benodigd behoeven. In dit deel worden de vragen geanalyseerd en beantwoord zodat er een duidelijkheid is in de materie en een goede basis wordt gelegd voor de daadwerkelijke implementatie beschreven in het volgende deel.

\section{Scope}
Het onderzoek zal zich beperken tot de benodigde informatie voor het implementeren van de nieuwe oplossing voor een geautomatiseerde SOUP analyse. Het zal ingaan op de gebruikte ontwikkelstack binnen Eaglescience en bestaande architectuur gezien de nieuwe oplossing een onderdeel is van een al bestaand project en hier dus naatloos op moet integreren.
Aan de orde zullen komen:
\begin{itemize}
\item Wat is een SOUP analyse?
  \begin{itemize}
  \item Welke oplossingen bestaan er op dit moment om een SOUP analyse te doen?
  \item Is een API / Database waar kwetsbaarheden in opgesomt zijn?
  \end{itemize}
\item Wat is de ontwikkel stack waar Eaglescience mee werkt?
\item Hoe ziet de portal er op dit moment uit en hoe is het nieuwe onderdeel hierin te integreren?
\end{itemize}

In feite zijn er twee onderzoeken die gedaan moeten worden. ten eerste is er een onderzoek naar de gebruikte middelen binnen eaglescience en daarnaast een theoretisch onderzoek naar termen binnen SOUP analyse.

In het komende hoofdstuk wordt de methode duidelijk hoe de onderzoeken gedaan worden.
en de hoofdstukken daarna zullen de vragen en de daarbij horende antwoorden beschrijven.
