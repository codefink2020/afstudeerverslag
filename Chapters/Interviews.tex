% Appendix A

\chapter{Interviews \& gesprekken}

%----------------------------------------------------------------------------------------
\section{Intake gesprek CTO over requirements en stakeholders}
\subsection{Doel}
Het doel van dit gesprek is het verkrijgen van duidelijkheid over requirements en de aanwijzing van andere stakeholders voor de module.
\subsection{Opzet}
Het gesprek heeft een open structuur waarbij er een leidraad is in de vragen die ik heb opgesteld voorafgaand aan het gesprek
\subsection{Verslag}
\textbf{Inleiding}
Aangegeven wat het doel is van het gesprek: requirements gathering en het vaststellen van stakeholders die in latere gesprekken geinterviewt kunnen worden over hun requirements. Afgesproken is ook dat er gesproken wordt in je en jij.\\
\textbf{Vraag1: Wat is de huidig situatie volgens jou, Hoe wordt er op dit moment een zorg gedragen dat de software die er gebouwd wordt veilig is voor productie?}
\lipsum[01]\\

\textbf{Vraag2: In de opdracht staat vermeld welke eisen er gesteld staan aan de module, hoe zie je de werkwijze in de toekomst ten opzicht van nu?}
\lipsum[03]\\

\textbf{Vraag4: Nu je terug kijkt op de opdracht die gegeven is, zijn er toevoegingen die nu, 4 weken na het uitbrengen van de opdracht, bestaan? Of zijn er zaken veranderd ten opzicht van inzichten die in de tussentijd zijn ontstaan.}
\lipsum[05]\\

\textbf{Vraag5: Welke Stakeholders zie jij voor dit project, wie heeft er het meeste nut van de nieuwe module? }
\lipsum[06]\\

\textbf{Vraag6: Wie zijn er op het moment bezig met de ontwikkeling van portal en kan ik inschakkelen als ik hulp nodig heb tijdens de implementatie?}
\lipsum[09]\\
\textbf{Vraag4: }
\lipsum[07]\\

\section{Interview Senior Developer t.b.v dev-stack onderzoek}
\subsection{Doel}
Het doel van dit onderzoek is het verkrijgen van meer informatie over de huidige dev-stack die gebruikt wordt door Eaglescience. En eventuele kennis over een bibliotheken waar kennis over is maar nooit gebuikt voor het implementeren van een automatische oplossing.
\subsection{Opzet}
Het gesprek is opgezet als een interview met open vragen die opgesteld zijn naar aanleiding van bevindingen in de requirements analyse.
\subsection{Verslag}
\textbf{Inleiding}
Aangegeven wat het doel is van het interview en dat het interview uit X vragen bestaat en dat we er een 45 minuten voor uit hebben getrokken.\\
\textbf{Vraag 1: Hoe wordt de analyse op dit moment uitgevoerd?}\\
\lipsum[01]\\
\\
\textbf{Vraag 2: Zijn er al pakketten / hulpmiddelen in gebruik?}\\
\lipsum[02]\\
\\
\textbf{Vraag 2a: Zo ja kunnen we deze integreren?}\\
\lipsum[03]\\
\\
\textbf{Vraag 2b: Is er al onderzoek gedaan door een medewerker naar hulpmiddelen. en wat was de reden dat deze nooit zijn geintegreerd in de huidige pipeline?}\\
\lipsum[04]\\
\\
\textbf{Vraag 3: blaat?}\\
\lipsum[05]\\

\subsection{Resultaat?}

\section{Stakeholder interview CTO / Dagelijks bestuur}
\subsection{Doel}
Eerder is er een intake gesprek geweest met de CTO over de opdracht hierin zijn de grote lijnen uitgezet over de module dit gesprek is bedoelt om meer inzicht te krijgen in de wensen van de CTO zelf.
\subsection{Opzet}

\subsection{Verslag}
\textbf{Vraag 1: blaat?}\\
\lipsum[01]\\
\\
\textbf{Vraag 2: blaat?}\\
\lipsum[02]\\
\\
\textbf{Vraag 3: blaat?}\\
\lipsum[03]\\
\\
\textbf{Vraag 4: blaat?}\\
\lipsum[04]\\
\\
\textbf{Vraag 5: blaat?}\\
\lipsum[05]\\

\section{Stakeholder interview Project manager}
\subsection{Doel}
\subsection{Opzet}
\subsection{Verslag}
\textbf{Vraag 1: blaat?}\\
\lipsum[01]\\
\\
\textbf{Vraag 2: blaat?}\\
\lipsum[02]\\
\\
\textbf{Vraag 3: blaat?}\\
\lipsum[03]\\
\\
\textbf{Vraag 4: blaat?}\\
\lipsum[04]\\
\\
\textbf{Vraag 5: blaat?}\\
\lipsum[05]\\

\section{Stakeholder interview Ontwikkelaar 1}
\subsection{Doel}
\subsection{Opzet}
\subsection{Verslag}
\textbf{Vraag 1: blaat?}\\
\lipsum[01]\\
\\
\textbf{Vraag 2: blaat?}\\
\lipsum[02]\\
\\
\textbf{Vraag 3: blaat?}\\
\lipsum[03]\\
\\
\textbf{Vraag 4: blaat?}\\
\lipsum[04]\\
\\
\textbf{Vraag 5: blaat?}\\
\lipsum[05]\\


\section{Stakeholder interview Ontwikkelaar 2}
\subsection{Doel}
\subsection{Opzet}
\subsection{Verslag}
\textbf{Vraag 1: blaat?}\\
\lipsum[01]\\
\\
\textbf{Vraag 2: blaat?}\\
\lipsum[02]\\
\\
\textbf{Vraag 3: blaat?}\\
\lipsum[03]\\
\\
\textbf{Vraag 4: blaat?}\\
\lipsum[04]\\
\\
\textbf{Vraag 5: blaat?}\\
\lipsum[05]\\



%----------------------------------------------------------------------------------------

\section{Appendix Section Test}
\lipsum[15]

\graffito{More dummy text}
\lipsum[16]

%----------------------------------------------------------------------------------------

\section{Another Appendix Section Test}
\lipsum[17]

\begin{table}
\myfloatalign
\begin{tabularx}{\textwidth}{Xll} \toprule
\tableheadline{labitur bonorum pri no} & \tableheadline{que vista}
& \tableheadline{human} \\ \midrule
fastidii ea ius & germano &  demonstratea \\
suscipit instructior & titulo & personas \\
\midrule
quaestio philosophia & facto & demonstrated \\
\bottomrule
\end{tabularx}
\caption[Autem usu id]{Autem usu id.}
\label{tab:moreexample}
\end{table}

\lipsum[18]

There is also a useless Pascal listing below: \autoref{lst:useless}.

\begin{lstlisting}[float=b,language=Pascal,frame=tb,caption={A floating example (\texttt{listings} manual)},label=lst:useless]
for i:=maxint downto 0 do
begin
{ do nothing }
end;
\end{lstlisting}
