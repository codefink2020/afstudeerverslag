% Appendix A

\chapter{Interviews \& gesprekken}

%----------------------------------------------------------------------------------------

\section{Opdrachtgever, opdracht en requirements analyse}
In deze appendix zijn verslagen van interviews en gesprekken te vinden die gevoerd zijn tijdens het onderzoek en de ontwikkeling van de nieuwe module.
Interviews en gesprekken die plaats hebben gevonden in het kader van de verduidelijking van de opdracht en opdrachtgever.

  \subsection{Intake gesprek CTO over requirements en stakeholders}
  \subsubsection{Doel}
Het doel van dit gesprek is het verkrijgen van duidelijkheid over requirements en de aanwijzing van andere stakeholders voor de module.
  \subsubsection{Opzet}
Het gesprek heeft een open structuur waarbij er een leidraad is in de vragen die ik heb opgesteld voorafgaand aan het gesprek
\subsubsection{Verslag}
\textbf{Inleiding}
Aangegeven wat het doel is van het gesprek: requirements gathering en het vaststellen van stakeholders die in latere gesprekken geinterviewt kunnen worden over hun requirements. Afgesproken is ook dat er gesproken wordt in je en jij.\\
\textbf{Vraag1: Wat is de huidig situatie volgens jou, Hoe wordt er op dit moment een zorg gedragen dat de software die er gebouwd wordt veilig is voor productie?}
\lipsum[01]\\

\textbf{Vraag2: In de opdracht staat vermeld welke eisen er gesteld staan aan de module, hoe zie je de werkwijze in de toekomst ten opzicht van nu?}
\lipsum[03]\\

\textbf{Vraag4: Nu je terug kijkt op de opdracht die gegeven is, zijn er toevoegingen die nu, 4 weken na het uitbrengen van de opdracht, bestaan? Of zijn er zaken veranderd ten opzicht van inzichten die in de tussentijd zijn ontstaan.}
\lipsum[05]\\

\textbf{Vraag5: Welke Stakeholders zie jij voor dit project, wie heeft er het meeste nut van de nieuwe module? }
\lipsum[06]\\

\textbf{Vraag6: Wie zijn er op het moment bezig met de ontwikkeling van portal en kan ik inschakkelen als ik hulp nodig heb tijdens de implementatie?}
\lipsum[09]\\
\textbf{Vraag4: }
\lipsum[07]\\

\subsubsection{Resultaat?}

\subsection{Interview met projectmanager als stakeholder van de nieuwe module}
\subsubsection{Doel}
Het doel van dit gesprek is het verkrijgen van duidelijkheid over requirements en de aanwijzing van andere stakeholders voor de module.
\subsubsection{Opzet}
Het gesprek heeft een open structuur waarbij er een leidraad is in de vragen die ik heb opgesteld voorafgaand aan het gesprek
\subsubsection{Verslag}
\textbf{Inleiding}
Aangegeven wat het doel is van het gesprek: requirements gathering en het vaststellen van stakeholders die in latere gesprekken geinterviewt kunnen worden over hun requirements. Afgesproken is ook dat er gesproken wordt in je en jij.\\
\textbf{Vraag1: Wat is de huidig situatie volgens jou, Hoe wordt er op dit moment een zorg gedragen dat de software die er gebouwd wordt veilig is voor productie?}
\lipsum[01]\\

\textbf{Vraag2: In de opdracht staat vermeld welke eisen er gesteld staan aan de module, hoe zie je de werkwijze in de toekomst ten opzicht van nu?}
\lipsum[03]\\

\textbf{Vraag4: Nu je terug kijkt op de opdracht die gegeven is, zijn er toevoegingen die nu, 4 weken na het uitbrengen van de opdracht, bestaan? Of zijn er zaken veranderd ten opzicht van inzichten die in de tussentijd zijn ontstaan.}
\lipsum[05]\\

\textbf{Vraag5: Welke Stakeholders zie jij voor dit project, wie heeft er het meeste nut van de nieuwe module? }
\lipsum[06]\\

\textbf{Vraag6: Wie zijn er op het moment bezig met de ontwikkeling van portal en kan ik inschakkelen als ik hulp nodig heb tijdens de implementatie?}
\lipsum[09]\\
\textbf{Vraag4: }
\lipsum[07]\\

\subsubsection{Resultaat?}


\subsection{Interview met (senior) developer als stakeholder van de nieuwe module.}
\subsubsection{Doel}
Het doel van dit gesprek is het verkrijgen van duidelijkheid over requirements en de aanwijzing van andere stakeholders voor de module.
\subsubsection{Opzet}
Het gesprek heeft een open structuur waarbij er een leidraad is in de vragen die ik heb opgesteld voorafgaand aan het gesprek
\subsubsection{Verslag}
\textbf{Inleiding}
Aangegeven wat het doel is van het gesprek: requirements gathering en het vaststellen van stakeholders die in latere gesprekken geinterviewt kunnen worden over hun requirements. Afgesproken is ook dat er gesproken wordt in je en jij.\\
\textbf{Vraag1: Wat is de huidig situatie volgens jou, Hoe wordt er op dit moment een zorg gedragen dat de software die er gebouwd wordt veilig is voor productie?}
\lipsum[01]\\

\textbf{Vraag2: In de opdracht staat vermeld welke eisen er gesteld staan aan de module, hoe zie je de werkwijze in de toekomst ten opzicht van nu?}
\lipsum[03]\\

\textbf{Vraag4: Nu je terug kijkt op de opdracht die gegeven is, zijn er toevoegingen die nu, 4 weken na het uitbrengen van de opdracht, bestaan? Of zijn er zaken veranderd ten opzicht van inzichten die in de tussentijd zijn ontstaan.}
\lipsum[05]\\

\textbf{Vraag5: Welke Stakeholders zie jij voor dit project, wie heeft er het meeste nut van de nieuwe module? }
\lipsum[06]\\

\textbf{Vraag6: Wie zijn er op het moment bezig met de ontwikkeling van portal en kan ik inschakkelen als ik hulp nodig heb tijdens de implementatie?}
\lipsum[09]\\
\textbf{Vraag4: }
\lipsum[07]\\

\subsubsection{Resultaat?}

\section{Onderzoek architectuur Eaglescience}
\subsection{Interview Senior Developer t.b.v dev-stack onderzoek}
\subsubsection{Doel}
Het doel van dit onderzoek is het verkrijgen van meer informatie over de huidige dev-stack die gebruikt wordt door Eaglescience. En eventuele kennis over een bibliotheken waar kennis over is maar nooit gebuikt voor het implementeren van een automatische oplossing.
\subsubsection{Opzet}
Het gesprek is opgezet als een interview met open vragen die opgesteld zijn naar aanleiding van bevindingen in de requirements analyse.
\subsubsection{Verslag}
\textbf{Inleiding}
Aangegeven wat het doel is van het interview en dat het interview uit X vragen bestaat en dat we er een 45 minuten voor uit hebben getrokken.\\
\textbf{Vraag 1: Binnen Eaglescience wordt er veel gebruikt gemaakt van Scala, wat is de voornaamste reden om dit te doen?}\\
\lipsum[01]\\
\\
\textbf{Vraag 2:}\\
\lipsum[02]\\
\\
\textbf{Vraag 2a: Zo ja kunnen we deze integreren?}\\
\lipsum[03]\\
\\
\textbf{Vraag 2b: Is er al onderzoek gedaan door een medewerker naar hulpmiddelen. en wat was de reden dat deze nooit zijn geintegreerd in de huidige pipeline?}\\
\lipsum[04]\\
\\
\textbf{Vraag 3: blaat?}\\
\lipsum[05]\\

\subsubsection{Resultaat?}

\subsection{Interview Project manager t.b.v tooling}
\subsubsection{Doel}
Het doel van dit interview is het verkrijgen van informatie over de beweegredenen om Jira en Confluence te gebruiken alsook de de beweegredenen om een project aan te pakken zoals we dat nu doen.
\subsubsection{Opzet}
Het gesprek is opgezet als een interview met open vragen en vervolg vragen naar aanleiding van de gevonden informatie in het werknemers handboek.
\subsubsection{Verslag}
\textbf{Inleiding}
Aangegeven wat het doel is van het interview en dat het interview uit X vragen bestaat en dat we er een 45 minuten voor uit hebben getrokken.\\
\textbf{Vraag 1: Hoe wordt de analyse op dit moment uitgevoerd?}\\
\lipsum[01]\\
\\
\textbf{Vraag 2: Zijn er al pakketten / hulpmiddelen in gebruik?}\\
\lipsum[02]\\
\\
\textbf{Vraag 2a: Zo ja kunnen we deze integreren?}\\
\lipsum[03]\\
\\
\textbf{Vraag 2b: Is er al onderzoek gedaan door een medewerker naar hulpmiddelen. en wat was de reden dat deze nooit zijn geintegreerd in de huidige pipeline?}\\
\lipsum[04]\\
\\
\textbf{Vraag 3: blaat?}\\
\lipsum[05]\\

\subsubsection{Resultaat?}

\subsection{Interview senior developer t.b.v tooling met Build en deploy specifiek}
\subsubsection{Doel}
Het doel van dit interview is het verkrijgen van informatie over de beweegredenen om Jira en Confluence te gebruiken alsook de de beweegredenen om een project aan te pakken zoals we dat nu doen.
\subsubsection{Opzet}
Het gesprek is opgezet als een interview met open vragen en vervolg vragen naar aanleiding van de gevonden informatie in het werknemers handboek.
\subsubsection{Verslag}
\textbf{Inleiding}
Aangegeven wat het doel is van het interview en dat het interview uit X vragen bestaat en dat we er een 45 minuten voor uit hebben getrokken.\\
\textbf{Vraag 1: Hoe wordt de analyse op dit moment uitgevoerd?}\\
\lipsum[01]\\
\\
\textbf{Vraag 2: Zijn er al pakketten / hulpmiddelen in gebruik?}\\
\lipsum[02]\\
\\
\textbf{Vraag 2a: Zo ja kunnen we deze integreren?}\\
\lipsum[03]\\
\\
\textbf{Vraag 2b: Is er al onderzoek gedaan door een medewerker naar hulpmiddelen. en wat was de reden dat deze nooit zijn geintegreerd in de huidige pipeline?}\\
\lipsum[04]\\
\\
\textbf{Vraag 3: blaat?}\\
\lipsum[05]\\

\subsubsection{Resultaat?}

\section{Onderzoek architectuur SOUP analyse}
\subsection{Interview Senior Developer t.b.v SOUP analyse}
\subsubsection{Doel}
Het doel van dit onderzoek is het verkrijgen van meer informatie over de huidige dev-stack die gebruikt wordt door Eaglescience. En eventuele kennis over een bibliotheken waar kennis over is maar nooit gebuikt voor het implementeren van een automatische oplossing.
\subsubsection{Opzet}
Het gesprek is opgezet als een interview met open vragen die opgesteld zijn naar aanleiding van bevindingen in de requirements analyse.
\subsubsection{Verslag}
\textbf{Inleiding}
Aangegeven wat het doel is van het interview en dat het interview uit X vragen bestaat en dat we er een 45 minuten voor uit hebben getrokken.\\
\textbf{Vraag 1: Binnen Eaglescience wordt er veel gebruikt gemaakt van Scala, wat is de voornaamste reden om dit te doen?}\\
\lipsum[01]\\
\\
\textbf{Vraag 2:}\\
\lipsum[02]\\
\\
\textbf{Vraag 2a: Zo ja kunnen we deze integreren?}\\
\lipsum[03]\\
\\
\textbf{Vraag 2b: Is er al onderzoek gedaan door een medewerker naar hulpmiddelen. en wat was de reden dat deze nooit zijn geintegreerd in de huidige pipeline?}\\
\lipsum[04]\\
\\
\textbf{Vraag 3: blaat?}\\
\lipsum[05]\\

\subsubsection{Resultaat?}

\subsection{Interview Project manager informatie voorziening }
\subsubsection{Doel}
Het doel van dit interview is het verkrijgen van informatie over de beweegredenen om Jira en Confluence te gebruiken alsook de de beweegredenen om een project aan te pakken zoals we dat nu doen.
\subsubsection{Opzet}
Het gesprek is opgezet als een interview met open vragen en vervolg vragen naar aanleiding van de gevonden informatie in het werknemers handboek.
\subsubsection{Verslag}
\textbf{Inleiding}
Aangegeven wat het doel is van het interview en dat het interview uit X vragen bestaat en dat we er een 45 minuten voor uit hebben getrokken.\\
\textbf{Vraag 1: Hoe wordt de analyse op dit moment uitgevoerd?}\\
\lipsum[01]\\
\\
\textbf{Vraag 2: Zijn er al pakketten / hulpmiddelen in gebruik?}\\
\lipsum[02]\\
\\
\textbf{Vraag 2a: Zo ja kunnen we deze integreren?}\\
\lipsum[03]\\
\\
\textbf{Vraag 2b: Is er al onderzoek gedaan door een medewerker naar hulpmiddelen. en wat was de reden dat deze nooit zijn geintegreerd in de huidige pipeline?}\\
\lipsum[04]\\
\\
\textbf{Vraag 3: blaat?}\\
\lipsum[05]\\

\subsubsection{Resultaat?}

\subsection{Interview senior developer t.b.v tooling met SOUP analyse specifiek}
\subsubsection{Doel}
Het doel van dit interview is het verkrijgen van informatie over de beweegredenen om Jira en Confluence te gebruiken alsook de de beweegredenen om een project aan te pakken zoals we dat nu doen.
\subsubsection{Opzet}
Het gesprek is opgezet als een interview met open vragen en vervolg vragen naar aanleiding van de gevonden informatie in het werknemers handboek.
\subsubsection{Verslag}
\textbf{Inleiding}
Aangegeven wat het doel is van het interview en dat het interview uit X vragen bestaat en dat we er een 45 minuten voor uit hebben getrokken.\\
\textbf{Vraag 1: Hoe wordt de analyse op dit moment uitgevoerd?}\\
\lipsum[01]\\
\\
\textbf{Vraag 2: Zijn er al pakketten / hulpmiddelen in gebruik?}\\
\lipsum[02]\\
\\
\textbf{Vraag 2a: Zo ja kunnen we deze integreren?}\\
\lipsum[03]\\
\\
\textbf{Vraag 2b: Is er al onderzoek gedaan door een medewerker naar hulpmiddelen. en wat was de reden dat deze nooit zijn geintegreerd in de huidige pipeline?}\\
\lipsum[04]\\
\\
\textbf{Vraag 3: blaat?}\\
\lipsum[05]\\

\subsubsection{Resultaat?}




%----------------------------------------------------------------------------------------

\section{Appendix Section Test}
\lipsum[15]

\graffito{More dummy text}
\lipsum[16]

%----------------------------------------------------------------------------------------

\section{Another Appendix Section Test}
\lipsum[17]

\begin{table}
\myfloatalign
\begin{tabularx}{\textwidth}{Xll} \toprule
\tableheadline{labitur bonorum pri no} & \tableheadline{que vista}
& \tableheadline{human} \\ \midrule
fastidii ea ius & germano &  demonstratea \\
suscipit instructior & titulo & personas \\
\midrule
quaestio philosophia & facto & demonstrated \\
\bottomrule
\end{tabularx}
\caption[Autem usu id]{Autem usu id.}
\label{tab:moreexample}
\end{table}

\lipsum[18]

There is also a useless Pascal listing below: \autoref{lst:useless}.

\begin{lstlisting}[float=b,language=Java,frame=tb,caption={A floating example (\texttt{listings} manual)},label=lst:useless]
System,out.println("Hello World!")

for i:=maxint downto 0 do
begin
{ do nothing }
end;
\end{lstlisting}
