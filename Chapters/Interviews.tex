% Appendix A

\chapter{Interviews}

%----------------------------------------------------------------------------------------
\section{Intake gesprek}
\subsection{Doel}
Het doel van het intake gesprek is het verkrijgen van meer informatie na het verkrijgen van de initi\"ele opdracht.
\subsection{Opzet}
Het gesprek is opgezet als een interview met open vragen die opgesteld zijn naar aanleiding van de schriftelijke opdracht. 
\subsection{Verslag}
\textbf{Inleiding}
Aangegeven wat het doel is van het interview en dat het interview uit X vragen bestaat en dat we er een 45 minuten voor uit hebben getrokken.\\
\textbf{Vraag 1: Hoe wordt de analyse op dit moment uitgevoerd?}\\
\lipsum[01]\\
\\
\textbf{Vraag 2: Zijn er al pakketten / hulpmiddelen in gebruik?}\\
\lipsum[02]\\
\\
\textbf{Vraag 2a: Zo ja kunnen we deze integreren?}\\
\lipsum[03]\\
\\
\textbf{Vraag 2b: Is er al onderzoek gedaan door een medewerker naar hulpmiddelen. en wat was de reden dat deze nooit zijn geintegreerd in de huidige pipeline?}\\
\lipsum[04]\\
\\
\textbf{Vraag 3: blaat?}\\
\lipsum[05]\\

\section{Stakeholder interview CTO / Dagelijks bestuur}
\subsection{Doel}
Eerder is er een intake gesprek geweest met de CTO over de opdracht hierin zijn de grote lijnen uitgezet over de module dit gesprek is bedoelt om meer inzicht te krijgen in de wensen van de CTO zelf.
\subsection{Opzet}

\subsection{Verslag}
\textbf{Vraag 1: blaat?}\\
\lipsum[01]\\
\\
\textbf{Vraag 2: blaat?}\\
\lipsum[02]\\
\\
\textbf{Vraag 3: blaat?}\\
\lipsum[03]\\
\\
\textbf{Vraag 4: blaat?}\\
\lipsum[04]\\
\\
\textbf{Vraag 5: blaat?}\\
\lipsum[05]\\

\section{Stakeholder interview Project manager}
\subsection{Doel}
\subsection{Opzet}
\subsection{Verslag}
\textbf{Vraag 1: blaat?}\\
\lipsum[01]\\
\\
\textbf{Vraag 2: blaat?}\\
\lipsum[02]\\
\\
\textbf{Vraag 3: blaat?}\\
\lipsum[03]\\
\\
\textbf{Vraag 4: blaat?}\\
\lipsum[04]\\
\\
\textbf{Vraag 5: blaat?}\\
\lipsum[05]\\

\section{Stakeholder interview Ontwikkelaar 1}
\subsection{Doel}
\subsection{Opzet}
\subsection{Verslag}
\textbf{Vraag 1: blaat?}\\
\lipsum[01]\\
\\
\textbf{Vraag 2: blaat?}\\
\lipsum[02]\\
\\
\textbf{Vraag 3: blaat?}\\
\lipsum[03]\\
\\
\textbf{Vraag 4: blaat?}\\
\lipsum[04]\\
\\
\textbf{Vraag 5: blaat?}\\
\lipsum[05]\\


\section{Stakeholder interview Ontwikkelaar 2}
\subsection{Doel}
\subsection{Opzet}
\subsection{Verslag}
\textbf{Vraag 1: blaat?}\\
\lipsum[01]\\
\\
\textbf{Vraag 2: blaat?}\\
\lipsum[02]\\
\\
\textbf{Vraag 3: blaat?}\\
\lipsum[03]\\
\\
\textbf{Vraag 4: blaat?}\\
\lipsum[04]\\
\\
\textbf{Vraag 5: blaat?}\\
\lipsum[05]\\



%----------------------------------------------------------------------------------------

\section{Appendix Section Test}
\lipsum[15]

\graffito{More dummy text}
\lipsum[16]

%----------------------------------------------------------------------------------------

\section{Another Appendix Section Test}
\lipsum[17]

\begin{table}
\myfloatalign
\begin{tabularx}{\textwidth}{Xll} \toprule
\tableheadline{labitur bonorum pri no} & \tableheadline{que vista}
& \tableheadline{human} \\ \midrule
fastidii ea ius & germano &  demonstratea \\
suscipit instructior & titulo & personas \\
\midrule
quaestio philosophia & facto & demonstrated \\
\bottomrule
\end{tabularx}
\caption[Autem usu id]{Autem usu id.}
\label{tab:moreexample}
\end{table}

\lipsum[18]

There is also a useless Pascal listing below: \autoref{lst:useless}.

\begin{lstlisting}[float=b,language=Pascal,frame=tb,caption={A floating example (\texttt{listings} manual)},label=lst:useless]
for i:=maxint downto 0 do
begin
{ do nothing }
end;
\end{lstlisting}
