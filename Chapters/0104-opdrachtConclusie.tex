\chapter{Conclussie}\label{ch:opdrachtconclussie}

De opdrachtgever is EagleScience, welke als opdracht "zoek een methode om geautomatiseerd en periodiek een SOUP analyse te doen op kwetsbaarheden" EagleScience wil dit graag ontwikkelen omdat men een groei in zowel projecten als personeel ziet aankomen. Daarnaast is er een proces gaande om de Software Lifecycle Management welke nu in de praktijk al wordt uitgevoerd in een beleid te gieten. Het is in deze context goed om te meten welke kwetsbaarheden er zijn binnen een ontwikkelde applicatie. Echter is het voorkomen beter. Door het gebruik van deze module komt er inzicht in deze kwetsbaarheden en wordt als het goed is  de vraag opgeworpen of het gebruik van deze bibliotheek nog wel gerechtvaardigd is.


