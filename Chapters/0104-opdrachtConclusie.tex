\chapter{Conclussie}\label{ch:opdrachtconclussie}

De opdrachtgever is EagleScience, een bedrijf gevestigd in Amsterdam Sloterdijk, dat voor klanten op project basis software ontwikkelt en dit indien gewenst ook kan hosten. Het bedrijf ziet een groei tegemoet in zowel personeel als het aantal simultaan te ontwikkelende projecten. En is daarom op zoek om zaken te kunnen automatiseren. Eén van deze taken is het analyseren van externe bibliotheken, ook Software of Unkown Provenance(SOUP) genoemd. Het wil dit op een manier doen dat het zonder veel aanpassingen in de huidige pipeline kan worden geïntegreerd en op een dusdanige manier resultaten geeft dat inzichtelijk wordt voor zowel de betreffende ontwikkelaar als de betreffende projectmanager/product owner. De wens is tevens dat de analyse periodiek wordt uitgevoerd om op die manier een up-to-date beeld te krijgen. Als laatst moet de nieuwe module uitreidaar zijn op het moment dat er nieuwe technieken worden gebruikt door EagleScience.

