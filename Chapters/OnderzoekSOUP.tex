% Chapter X

\chapter{Onderzoek: SOUP analyse} % Chapter title

\label{OndSOUPAnalyse} % For referencing the chapter elsewhere, use \autoref{ch:voorOnderzoek}
In het vorige hoofdstuk is te lezen dat Eaglescience gebruik maakt van de volgende technologi\"en
\begin{itemize}
  \item Scala 2.XX
  \item TypeScript
  \item Jenkins
  \item Docker
  \item Azure cloud
\end{itemize}
Dit onderzoek richt voornamelijk op kwetbaarheden in software en de bestrijding ervan. En specifiek op de bovenstaande technieken. De hoofdvraag voor dit hoofdstuk luid dan ook: "Met welke kwetsbaarheden hebben we te maken binnen Eaglescience en hoe kunnen we deze opsporen op een geautomatiseerde manier zonder de huidige werkwijze te verstoren?" Uit deze hoofdvraag onstaan de volgende deelvragen die in dit onderzoek beantwoord worden met daarna een conclusie op de hoofdvraag.

\begin{itemize}
\item Welke soorten kwetsbaarheden zijn er?
\item Hoe kunnen deze kwetbaarheden hun weg vinden in onze gebouwde software?
\item Zijn er instanties die bijhouden waar zich kwetsbaarheden schuilhouden?
\item Wat zijn methodes om te onderzoeken of er in de bestaande software kwetbaarheden bevinden?
\item Is er een mogelijkheid om een third-party pakket in te zetten om dit te doen?
\end{itemize}


\section{Welke soorten kwetsbaarheden zijn er?}
Een instantie dat zich bezighoud met het bestuderen en onderhouden van kwetbaarheden binnen software is OWASP. (OWASP.org) Wat een nonprofit organisatie is dat zich bezig houd met het verbeteren van de veiligheid van software. Het bestaat uit verschillende communities in veel landen en heeft meer dan tientallen duizend leden. OWASP zorgt voor erkenning door training en educatie aan te bieden als ook Tools en resources, maar vooral de community en het netwerk is van groot belang. Een van de zaken die zei doen is het opstellen van een top10 die eens in de 5 jaar wordt geupdate\footnote{Helaas is de laatste versie die aan het einde van 2021 uit moet komen nog niet beschikbaar op het moment van schrijven}. De OWASP-Top10 wordt samengesteld uit data van meer dan 100.000 productie applicaties en APIs wat door meer dan 500 mensen is getest door 40 verschillende bedrijven. De top 10 is een aggegratie van deze data in de meest voorkomende issues met inachtneming van exploitabity, detectability en impact.
\begin{itemize}
\item \textbf{A01:2017} Injection
\item \textbf{A02:2017} Broken Authentication
\item \textbf{A03:2017} Sensitive Data Exposure
\item \textbf{A04:2017} XML External Entities (XXE)
\item \textbf{A05:2017} Broken Acces Control
\item \textbf{A06:2017} Security Misconfiguration
\item \textbf{A07:2017} Cross-Site Scripting (XSS)
\item \textbf{A08:2017} Insecure Deserialization
\item \textbf{A09:2017} Using components with Known vulnerabilities
\item \textbf{A10:2017} Insuffivient Logging \& Monitoring
\end{itemize}

Zoals te zien is in de top 10 staat op nummer A09 "Using components with Known vulberabilities"
\section{Hoe kunnen deze kwetbaarheden hun weg vinden in onze gebouwde software?}
\section{Zijn er instanties die bijhouden waar zich kwetsbaarheden schuilhouden?}
\section{Wat zijn methodes om te onderzoeken of er in de bestaande software kwetbaarheden bevinden?}
\section{Is er een mogelijkheid om een third-party pakket in te zetten om dit te doen?}




\section{Wat is Software of Unkown Pedigree(SOUP)?}
Volgens Wikipedia is \'software of unkown pedigree\' software dat niet volgens een software ontwikkel process of methode is ontwikkeld dat bekend is bij de eindgebruiker alsook software dat onbekende veiligheids eigenschappen heeft. De term wordt vooral gebruikt binnen het ontwikkelen van medische software.\\  % Bron: https://en.wikipedia.org/wiki/Software_of_unknown_pedigree
Door het gebruik van dit soort software kan men er dus nooit van uitgaan dat het veilig is.  SOUP komt veelvuldig voor in Open-source software, echter is closed software ook nooit gegarandeerd 100\% veilig als het van derden komt. Betekende dat eigen software niet altijd 100\% veilig is maar er is wel meer controle dan als er software van derden gebruikt wordt.

Deze definitie geld voor zowel volledige software pakketten als voor bibliotheken. Het gebruik van bibliotheken is iets waar Eaglescience veelvuldig gebruik van maakt. En geeft dit mogelijk problemen bij de veiligheid van de software zelf.

\section{Welk type bibliotheken kunnen onbekende veiligheid eigenschappen hebben of is de methode van ontwikkelen niet bekend?}
In principe is iedere bibliotheek waarvan niet de ontwikkelmethode of process te herleiden is, software of unkown pedigree. Veelal zijn dit de open-source bibliotheken die als onderdeel worden gebruikt in het ontwikkelen van zowel de frontend als de backend.

\subsection{OpenSource Software}Open-Source software staat er om bekend dat het veelal door een community wordt ontwikkeld waar bij de structuur niet altijd direct zichtbaar is en daarmee dus ook de methode niet altijd duidelijk.  	Eric S. Raymond spreek ook wel van een Bazaar model % Bron:https://en.wikipedia.org/wiki/Open-source_software#Development_model %bron : https://en.wikipedia.org/wiki/The_Cathedral_and_the_Bazaar
waarbij iedereen toegang heeft tot de source code en er zijn eigen aanpassingen aan doet, dit model stelt ook dat software snel gereleased wordt en er daarna frequente iteraties zijn om de software up-to-date te houden. Soms metd egebruiker als mede ontwikkelaar. Het grote voordeel is dat er meerdere inzichten van veel developers kunnen worden benut. dit geeft gelijk ook het nadeel dat het er vaak op neer komt dat er meerdere methodes gebruikt kunnen worden en daarmee dus niet kan worden vastgesteld hoe een pakket is ontwikkeld.

\subsection{Closed Source Software}% Bron:https://en.wikipedia.org/wiki/Proprietary_software
Closed source software is software dat is gebouwd onder een licentie, dit wil zeggen dat de gebruiker van de software/ bibliotheek meestal een bedrag moet betallen om het te betalen. Enkele voordelen van closed software is dat je weet wie het onwikkeld heeft en na een beetje onderzoek ook hoe de sotware gebouwd is. Al is dit laatste niet geheel relevant gezien er meestal een binary wordt geleverd zonder dat er aanpassingen vanuit de gebruiker in de source kan worden gedaan. Een nadeel is dat er meestal voor de software moet worden betaald en daarom dus duurder kan worden om het te gebuiken.

\subsection{Conclusie}

% tabel toevoegen https://en.wikipedia.org/wiki/Proprietary_software Types
het verschil tussen open en closed software zit hem vooral op de manier waarop het vervaardigd is, in een community of door een bedrijf. daarnaast zijn de kosten ook een verschil omdat er bij closes-source altijd een licentie moet worden verkregen en hier vaak (herhaaldelijke kosten) aan zitten. is dit bij open-source meestal
niet en wordt er vaak een vrijwillige bijdrage geleverd voor het gebruik ervan.

op basis van de kosten zou je kunnen zeggen dat open-source goedkoper is dan closed source bij de aanschaf. echter moet er rekening gehouden worden dat er niet altijd bekend is heo de software gebouwd is en wat de potenti\"ele gevaren hiervan zijn.

\section{Hoe wordt er op het dit moment een SOUP analyse uitgevoerd door Eaglescience en wat zijn de resource die gebruikt worden?}

Interview bouwen....


\lipsum[10]


\section{deelvraag 2}

\lipsum[10]

\section{deelvraag 3}

\lipsum[10]

\section{deelvraag 4}

\lipsum[10]
\section{deelvraag 5}

\lipsum[10]


\section{deelvraag 6}
\lipsum[10]
