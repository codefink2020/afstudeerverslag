\chapter{Onderzoeksmethode}\label{ch:onderzoeksmethode} % Chapter title

Op basis van de requirements analyse beschreven in het vorige hoofdstuk zijn er een aantal vragen ontstaan die verder onderzoek behoeven. Er is gekozen om het vooronderzoek op te delen in drie delen, waarbij iedere deel in een eigen hoofdstuk wordt beschreven. De onderwerpen zijn gekozen omdat ze kennis verschaffen in de materie wat vervolgens gebruikt kan worden in het uiteindelijke ontwerp.
Als eerste dient er gekeken worden hoe er binnen EagleScience software ontwikkeld en uitgerold wordt. Deze kennis is nodig om te zien hoe er op dit moment een applicatie wordt uitgerold en/of er aanpassingen gedaan moeten worden binnen die methode. Ook geeft het inzicht in de gebruikte dev-stack welke op zijn beurt weer beperkingen oplegt in de mogelijk te gebruiken tools. Naast vragen binnen EagleScience dient er ook een onderzoek gedaan te worden over het fenomeen SOUP en de gevaren die het mogelijk met zich meebrengt. Met daarop volgend een onderzoek naar tooling die ingezet kan worden om kwetsbaarheden binnen SOUP op te sporen binnen de huidige manier van uitrollen.

Dit hoofdstuk beschrijft een plan van aanpak hoe de onderzoeken gedaan gaan worden met in het bijzonder aandacht voor het doel, scope en de methode die gehanteerd wordt om het gezette doel te bereiken.

\section{Scope}\label{sec:Scope}
Het veiliger maken van software is een breed gebied met veel relevante onderwerpen. Deze opdracht beperkt zich tot het analyseren van bibliotheken van derden (SOUP-bibliotheken). Het voor onderzoek dat gedaan moet worden zal zich dan ook beperken tot deze analyse. Daarnaast is het noodzaak om alleen een onderzoek te doen naar methode die passen binnen de huidige manier van softwareontwikkeling binnen EagleScience. Dit geeft dus een beperking in het onderzoek naar de beste methode om kwetsbaarheden te zoeken in software.
[Garage]
Software veiligheid is een breed gebied met veel relevante onderwerpen binnen het ontwikkelen van software.
Het onderzoek zal zich beperken tot de benodigde informatie voor het implementeren van de nieuwe oplossing voor een geautomatiseerde SOUP-analyse. Hierdoor zullen andere onderwerpen binnen het veilig ontwerpen van software niet worden onderzocht. Zo zullen ook niet alle bedrijfprocessen binnen EagleScience worden onderzocht en alleen gekeken worden naar de processen binenn het uitrollen van de software. EagleScience doet veel aan het veilig ontwerpen van zijn eigen ontwikkelde software waardoor de noodzaak op het moment van het geven van de opdracht niet hoog ligt.
[Garage]
\section{Vooronderzoeken}\label{sec:vooronderzoeken}
De vooronderzoeken hebben ieders een eigen onderwerp, doel, onderzoeksvraag en methode
\subsection{Onderzoek 1: Architectuur binnen EagleScience}\label{subsec:onderzoeksmethode-architectuur-binnen-eaglescience}
Het \textbf{doel} van dit onderzoek is om kennis te vergaren over de manier van uitrollen van software binnen EagleScience. Daarnaast moet het onderzoek inzicht geven in de gebruikte dev-stack( programeertalen, tooling en frameworks). De uitkomst is relevant omdat het een basis is waar de nieuwe oplossing onderdeel van wordt. De \textbf{Scope} van dit onderzoek beperkt zich dan ook alleen op de processen van die direct iets te maken hebben met het ontwikkelen en uitrollen van de software. Uit het doel en de scope komt de volgende ~\textbf{onderzoeksvraag}: "Welke methode en ontwikkeltalen worden er gebruikt om binnen EagleScience software te ontwikkelen en uit te rollen?". De \textbf{methodes} die gebruikt worden zijn de interne documenten over de verschillende platformen en werkwijzen die beschikbaar zijn gesteld door het bedrijf zelf. De kennis uit deze documenten kan worden aangevuld door informatie dat uit bronnen van leveranciers afkomstig is. Daarnaast zullen er gesprekken plaats vinden met collega's. Deze input geeft het volgende onderzoeks model wat te zien is in figuur~\ref{fig:OnderzoeksModelEaglescience}
De \textbf{bronnen} die voornamelijk gebruikt worden zijn interne documenten waarin vermeld staat hoe een project verloopt en welke tooling er gebruikt wordt. Daarnaast zal er veel gebruik gemaakt worden van kennis van collega's en die ik zelf heb opgedaan dan al niet tijdens mijn opleiding.
\begin{figure}[htbp]
    \myfloatalign
    \includegraphics[width=10cm]{gfx/OnderzoeksmodelES}
    \caption{Onderzoeksmodel Eaglescience}
    \label{fig:OnderzoeksModelEaglescience}
\end{figure}


%\newpage % quickfix om volgordelijkheid te veranderen voor figuren...


\subsection{Onderzoek 2: Literatuur studie veiligere software door SOUP analyse}\label{subsec:onderzoek-literatuur-studie-soup}
Het \textbf{doel} van dit onderzoek is om inzicht te krijgen wat een SOUP-analyse is en hoe relevant het is om dit te doen. Daarnaast wordt er gekeken wat de SOUP-analyse toevoegd aan de veiligheid van de software die EagleScience levert. Met dit doel is de \textbf{scope} dat er alleen gekeken wordt naar soup-analyses en dat andere methoden die software veiliger maken wel aanbod komen als referentie. Maar niet verder worden uitgezocht.  De \textbf{onderzoeksvraag} luid: "Hoe kan SOUP op een effectieve manier worden geanalyseerd en hoe maakt dit software veiliger?". De gebruikte \textbf{methodes} zullen deskresearch zijn aangevuld met interviews. Interviews met collega's zullen inzicht brengen in methodes die al worden toegepast om software veiliger te maken. Daarnaast zijn een aantal conferenties die gehouden worden met software veiligheid als onderwerp. Deze conferenties staan nu door de huidige wereld situatie online en zijn vaak terug te kijken. Het onderzoeksmodel welke te zien is in figuur~\ref{fig:OnderzoeksModelNoodZaakSOUP} geeft een beeld hoe de methoden worden toegepast om tot een eindresultaat te komen. De \textbf{bronnen} die gebruikt zullen worden zullen online bronnen zijn aangevuld met vraaggesprekken met collega's en medewerkers van instanties die zich bezighouden met het veiliger maken van software. Daarnaast worden relevante presentaties van conferenties terug gekeken voor verdere informatie.
\begin{figure}[htbp]
    \myfloatalign
    \includegraphics[width=10cm]{gfx/OnderzoeksmodelSOUP}
    \caption{Onderzoeksmodel SOUP analyse module}
    \label{fig:OnderzoeksModelNoodZaakSOUP}
\end{figure}

%
%Dit onderzoek is vooral bedoeld om wegwijs te raken in de wereld het zoeken naar kwetsbaarheden binnen externe bibliotheken.
%Het gaat voornamelijk in op de betekenis van de verschillende begrippen en vervolgens hoe belangrijk het is om deze analyse uit te voeren.
%De uitkomst van dit onderzoek is een basis kennis die als entree voor de komende onderzoeken gebruikt kan worden.
%Het onderzoek heeft niet echt een hoofdvraag waardoor er een duidelijke scope moet worden gedefineerd.
%De volgende zaken moet duidelijk worden in dit onderzoek:
%\begin{itemize}
%  \item "Wat is SOUP?"
%  \item "Waarom kan het gebruik van SOUP gevaarlijk zijn?"
%  \item "Hoe worden deze gevaren/kwetsbaarheden gelogd?"
%  \item "Wat is een CVE en een CVSS?"
%\end{itemize}
%
%
%

\newpage % tijdelijk om te zorgen dat figuur op zelfde pagina als tekst komt
\subsection{Onderzoek 3: Implementatie van een SOUP-analyse}\label{subsec:onderzoek-naar-soup-analyse}
Het \textBF{doel} is om een tooling/bibliotheken te vinden die gebruikt kan worden om soup analyses te doen die binnen de huidige methode van uitrollen van eaglescience past. Daarnaast moet in kaart worden gebracht welke output deze methode genereert zodat het input geeft in de nieuwe oplossing. De \textbf{methode} die gebruikt wordt is deskresearch om te onderzoeken welke tooling er bestaat. Als er een selectie is gemaakt voor een tool dient deze in een kleine testopstelling getest te worden om vervolgens te kijken of we deze kunnen implementeren in de bestaande uitrolmethode. De \textbf{bronnen} die gebruikt zullen worden zijn informatie bronnen van leveranciers van dergelijke tooling. Daarnaast zullen de bevindingen middels een review worden geverifieerd op bruikbaarheid bij de opdrachgever. Het onderzoeksmodel voor dit onderzoek is te vinden in figuur~\ref{fig:OnderzoeksModelSOUPmethode}

\begin{figure}[htbp]
    \myfloatalign
    \includegraphics[width=10cm]{gfx/OnderzoeksModelSOUPMethode}
    \caption{Onderzoeksmodel SOUP-analyse module}
    \label{fig:OnderzoeksModelSOUPmethode}
\end{figure}
