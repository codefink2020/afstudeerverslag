% Todo Betere titel vinden
\chapter{Onderzoek: gebruik externe bibliotheken, het gevaar en hoe veiliger te maken}\label{ch:externeBibliothekengebruikGevaren}
Om een houvast te hebben in het ontwikkelen van een methode voor het analyseren van externe bibliotheken op kwetsbaarheden moet er een onderzoek gedaan worden over de theorie over het gebruik van externe bibliotheken, het gevaar en op welke manier er veilig mee gewerkt kan worden. De onderzoeksvraag voor dit onderzoek luidt dan ook: Wat is het effect van het gebruik van externe bibliotheken bij het ontwikkelen van software, welke gevaren brengt dit met zich mee en wat kan er gedaan worden om deze gevaren te minimaliseren?". Deze onderzoeksvraag is op te delen in de volgende deelvragen die ieders in een sectie worden beantwoord met als doel een conclussie te kunnen geven die als input te gebruiken is voor het onderzoek naar een methode voor SOUP analyses binnen EagleScience.

\begin{itemize}
    \item Hoeveel gebruik wordt er gemaakt van externe bibliotheken bij de ontwikkeling van applicaties?
    \item Waarom worden externe bibliotheken gebruikt in het ontwikkelen van software?
    \item Wat zijn potienteel gevaren die het gebruik van externe bibliotheken kunnen introduceren?
    \item Hoe kan er voorkomen worden dat er kkwetsbaarheden onstaan in een applicatie die gebruik maakt van externe bibliotheken?
\end{itemize}

\section{Hoeveel gebruik wordt er gemaakt van externe bibliotheken bij de ontwikkeling van applicaties?}\label{sec:hoeveel-gebruik-wordt-er-gemaakt-van-externe-bibliotheken-bij-de-ontwikkeling-van-applicaties?}
Software wordt tegenwoording voor een groot deel vervaardigd door gebruik te maken van open source\-code. Volgens een onderzoek gedaan door Synopsys\footnote{Synopsys is een bedrijf dat zich bezig houd met de ontwikkeling en verification van semiconductoren. Daarnaast ontwikkeld het tools voor verschillende taken in het domein software veiligheid. } (https://www.synopsys.com/software-integrity/resources/analyst-reports/open-source-security-risk-analysis.html?intcmp=sig-blog-ossra1) bestond in 2020 98\% van de 1546 geanalyseerde codebases uit open source componenten. Daarnaast werd er gevonden dat 84\% van deze codebases minimaal één kwetsbaarheid bevat met een gemiddelde van 158 kwetsbaarheden. De gemiddelde kwetsbaarheid was 2.2 jaar oud. Een ander belangrijk signalement dat Synopsys zag was dat er steeds meer codebases ontstonden wat minstens één kwetsbaarheid had.

Dit onderzoek gedaan door een bedrijf wat zelf tools verkoopt om kwetsbaarheden op te sporen en dus de cijfers in hun voordeel zijn afgerond. Is het probleem van kwetsbaarheden in bibliotheken van derden zeer groot. Als we onderzoeken van andere bedrijven bekijken komen die op een ongeveer dezelfde getallen. TideLift
\footnote{TideLift iseen bedrijf dat zich inzet voor verbeteren van het gebruik en veilig gebuik van OpenSource software(https://blog.tidelift.com/open-source-is-everywhere-survey-results-part-1)}
die meer onderzoek heeft gedaan over hoe een ontwikkelaar tegenover open-source componenten staan. Dit onderzoek wees uit dat ontwikkellaars over het algemeen voor Open-source kiezen dan voor betaalde software. Vaak om dat het ze de volgende eigenschappen verschaft:  flexibiliteit, snellere ontwikkeltijden, tevredenheid van de ontwikkelaar, en het kostenplaatje is minder. Ook wees het onderzoek uit dat de selectie voor het gebruik van een open-source project vaak gedaan werd op basis van hoe actief een project, hoe betrouwbaar de bron, en hoeveel mensen er aan het project meewerken.

Het onderzoek van SonarType\footnote{Sonartype is het bedrijf dat achter SonarQube zit( software voor het onderzoeken van source kwaliteit vooral ingezet op Java Projecten)} in 2021 gaf aan dat dat er een 650\% jaar op jaar steiging was in het aantal aanvallen gepleegd middels kwetsbaarheden in open-source bibliotheken.
[NOTE:] Nog uitbreiden met cijfers.

Ook al zijn de cijfers hierboven beschreven afkomstig van afhankelijke bronnen, ze geven wel aan dat er veel externe bibliotheken worden gebruikt omdat het vaak makkelijk, snel en een mogelijke flexibele oplossingen bied.

https://www.zdnet.com/article/its-an-open-source-world-78-percent-of-companies-run-open-source-software/

\section{Waarom worden externe bibliotheken gebruikt in het ontwikkelen van software?}\label{sec:waarom-worden-externe-bibliotheken-gebruikt-in-het-ontwikkelen-van-software?}
Bibliotheken zorgen over het algemeen ervoor dat herhalende taken die applicaties uit moeten voeren eenmalig wordt ontwikkeld en vervolgens voor meerdere projecten/applicaties in te zetten is. Op deze manier wordt ontwikkeltijd gewonnen omdat herhalende source-code geimporteerd kan worden en niet opnieuw geschreven hoeft te worden. Doordat het op een enkele plaats gedefineerd staat hoeft dit ook in het geval van een aanpassing, maar op een enkele plek aangepast te worden Dit scheeld wederom tijd, maar ook potentieel fouten omdat de aanpassing niet overal hoeft te worden doorgevoerd. Het gebruik van bibliotheken is dus bijna niet weg te denken gezien de druk die op het ontwikkelen van software staat. Een stap verder is het gebruik van externe bibliotheken die ervoor zorgen dat er om dezelfde redenen als hierboven nog minder ontwikkeltijd nodig is omdiezelfde taken te kunnen gebruiken echter wordt de bibliotheek niet door ontwikkelaars zelf ontwikkeld maar door een externe partij. Naast dat het gebruik van externe bibliotheken tijdswinst en dus de kosten voor het ontwikkelen van een applicatie kan verminderen zijn er nog een aantal voordelen die het benoemen waard zijn op het moment dat er veel gebruikte externe bibliotheken gebruikt worden in het ontwikkelproces: Als een bibliotheek gebruikt wordt door een groot aantal andere bedrijven ontstaat hier veel kennis over, wat kan leiden tot veel documentatie en support en een door de comunity ondersteunde standaard wat op zijn beurt weer helpt in het makkelijker op kunnen leiden van nieuw personeel. Daarnaast is er het feit dat als er door een 'grotere' community een bibluitheek gebruikt wordt deze ook vaker getest zal zijn, waarmee de functionele betrouwbaarheid verhoogd wordt.

\section{Wat zijn potienteel gevaren die het gebruik van externe bibliotheken kunnen introduceren?}\label{sec:wat-zijn-potienteel-gevaren-die-het-gebruik-van-externe-bibliotheken-kunnen-introduceren?}
Hoewel de voordelen die hierboven beschreven zijn enorm kunnen zijn voor bedrijven die software ontwikkelen. Zijn er ook een aantal nadelen te benoemen waarvan de grootste wel is dat er potentieel kwetsbaarheden kunnen worden geintroduceert in de applicatie die niet direct kunnen worden gezien door ontwikkelaars die de bibliotheken gebruiken. Deze kwetsbaarheden kunnen zich op meerdere manieren manifesteren in de vorm van een "Supply Chain Attack". Waarbij zoals de naam al doet vermoeden er een aanval plaats vindt middels software die zich in de dependencies bevindt deze aanval kan op verschillende manieren plaatsvinden, maar hebben meestal als doel dat er op een één of andere manier data bemachtigd wordt of anders de functionaliteit van de doelapplicatie op een dusdanige manier wordt aangetast dat deze in het voordeel is van de aanvallers. Een probleem met dit soort aanvallen is vaak dat deze lange tijd onopgemerkt kunnen blijven (SolarWinds attack) omdat externe bibliotheken niet vaak worden gecheckt op juistheid. En op het moment dat ze al opgemerkt worden en er een verslag is gemaakt in een Vulnerablitiy database (NIST, NVD, Mitre) Deze ook nog door andere bedrijven die dezelfde bibliotheek met dezelfde versie wordt gebruikt waarbij deze bedrijven ook kwetsbaar zijn. Uit onderzoek van Sonartype blijkt dat een groot deel van de applicaties niet worden bijgewerkt of vaak niet op de juiste manier zodat er altijd een kwetsbaarheid zal blijven bestaan.

\section{Hoe kan er voorkomen worden dat er kkwetsbaarheden onstaan in een applicatie die gebruik maakt van externe bibliotheken?}\label{sec:hoe-kan-er-voorkomen-worden-dat-er-kkwetsbaarheden-onstaan-in-een-applicatie-die-gebruik-maakt-van-externe-bibliotheken?}
Hoewel er veel bedrijven en instanties zich verantwoordelijk voelen, dan al niet voor winst, om het mogelijk te maken om vilige software te ontwikkelen. Bestaat de OWASP puur en alleen om dit te doen op een vrijwillige basis. Zei doen dit door awareness te kweken over het veilig ontwikkelen van software. Eén van de belangrijkste methoden die zij hebben is de OWASP top 10 waarin iedere 5 jaar een lijst wordt gepresenteerd met de meest ktitsiche aspecten voor het ontwikkelen van veilige software. In 2021 stond op plaats A06:2021 een item over Vulnerable and Outdated Components. Een item wat het onderzoek naar een methode en tools kan helpen Volgens dit item is de software kwetsbaar als er aan één van de volgende items kan worden voldaan: (vrij vertaald uit het origineel document https://owasp.org/Top10/A06\_2021-Vulnerable\_and\_Outdated\_Components/)
\begin{itemize}
    \item Als je niet alle versies van de gebruikte depenedencies weet (zowel van de client als server side). hiermee worden zowel de directe als de geneste dependencies bedoelt.
    \item Als de gebruikte componenten zelf kwetsbaar zijn, out-of-date zijn of niet meer worden ondersteuned. Dit geld voor alle componenten zoals OS, web/application Servers, Database management systemen, afhankelijke applicaties, API's (met zijn componenten), runtime omgevingen, en bibliotheken.
    \item Als er niet regelmatig gescanned wordt op kwetsbaarheden en er geen abonement is op de security bulletins van de gebruikte componenten.
    \item Als het onderliggende platform, framework, dependency niet gefixed of geupgrade wordt op een periodieke manier als ook op het moment dat er een kwetsbaarheid is gevonden. Dit komt vaak voor als een platform periodiek (eens per maand, eens per kwartaal) wordt geupdate als onderhoud. Vaak kan het voorkomen dat een systeem lekken heeft die pas laat worden gedicht.
    \item Als ontwikkelaars niet de compatibiliteit testen van geupdate, geupgrade en gepatchde bibliotheken.
    \item Als de componenten niet veilig zijn geconfigureerd. (zie A05:2021- Security Misconfiguration)
\end{itemize}
