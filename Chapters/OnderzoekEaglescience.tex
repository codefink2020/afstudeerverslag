% Chapter X

\chapter{Onderzoek: Architectuur binnen Eaglescience} % Chapter title

\label{OnderzoekArchituur} % For referencing the chapter elsewhere, use \autoref{ch:voorOnderzoek}
Dit onderzoek is een onderzoek binnen Eaglescience naar gebruikte technologi\"en welke dev-stack er gebruikt wordt. Ook zal er worden onderzocht welke build pipeline er gebruikt wordty en of mogelijkheden zijn om deze aan te passen om een SOUP-analyse uit te kunnen voeren.an SOUP-Bibliotheken mogelijk kan maken.
\section{Onderzoeksvraag}
De onderzoeksvraag luid: "Wat gebruikt Eaglescience aan tooling en dev-stack om dagelijks software te ontwikkellen. "
Uit deze onderzoeksvraag komen een aantal deelvragen die verder in dit onderzoek zullen worden beantwoord.
\begin{itemize}
  \item Hoe wordt op dit moment gewerkt binnen Eaglescience en dan met name op het gebied van Builds en het analyseren van SOUP-Bibliotheken?
  \item Welke Ontwikkeltalen gebruiken we binnen Eaglescience?

  \item Welke frameworks worden er gebruikt binnen de ontwikkeltalen?

  \item Hoe wordt op dit moment de ontwikkelde software gedeployed?
  \item Welke architectuur wordt er op dit moment gebruikt in de portal?
  \item Waar wordt de software uiteindelijk gedeployed?
  \item Welke methoden zijn er buiten Eaglescience om te zien of een bibliotheek kwetsbaarheden bevat?
  \item Wat is Software of Unkown Pedigree?
\end{itemize}


\section{Interview Senior Ontwikkelaar}
Om antwoorden te krijgen op de vraag welke dev-stack er gebruikt wordt bij Eaglescience is er een interview gehouden met een senior ontwikkelaar.Een verslag van het dit gesprek is te vinden in bijlage X%nog toewijzen.


\section{Hoe wordt er op het moment gewerkt binnen eagleScience?}
\subsection{projecten}
Projecten lopen binnen eaglescience een vooraf bepaalde fasen door die hieronder worden beschreven. De module die ontwikkeld wordt zal middels het zelfde principe worden ontwikkeld echter dan met veel interne kenmerken.
\subsubsection{Phase1: Sales \& acquisition}
Deze fase is een onderzoeksfase waarin vooral de wensen van de klant in kaart wordt gebracht. Hierbij kan gedacht worden aan de volgende zaken: Doel van de applicatie met daarbij de requirements en restraints, planning, het te gebruiken budget. Deze fase wordt vaak door de CEO, Senior Dev, en een projectmanager uitgevoerd. Het resultaat is een inschattingsdocument.
\subsubsection{Phase2: Project initiation}
In deze fase wordt het project opgestart. Er wordt een team samengesteld die gaat werken aan het ontwikkelen van de software. Er wordt een confluence en Jira pagina aangemaakt waarbij zaken over het project gedocumenteerd wordt. Deze fase is ook cruciaal om alle platform in gereedheid te brengen. te denken aan rechten voor de ontwikkelaars op Azure Cloud, Sentri, Jenkins en dergelijke. Als alles in gereedheid wordt gebracht is er een project kick-off waarbij hetteam wordt ingelicht over het project en taken die vervult moeten gaan worden.
\subsubsection{Phase3: Start \& Execution}
Dit is een iteratieve fase die in sprints doorloopt tot het project gereed is. Waarbij na iedere sprint een demo wordt gegeven.
\subsubsection{Phase4: Project Warp up}
In deze fase wordt het project opgeleverd aan de klant en wordt dan al niet door Eaglescience gehost op Azure cloud. Er wordt een project retro gehouden waarbij het team terugkijkt op de werkzaamheden en hoe deze verliepen, daarnaast is er een evaluatie met de klant.
\subsubsection{Phase5: Project maintenance}
Geen enkel software wordt direct zonder bugs opgeleverd deze fase duurt dan ook zolang de als software lifecycle is. of tot het budget van de klant op is :D In deze fase wordt support geleverd door Eaglescience op de source code en mogelijk aanpassingen gedaan om bugs te verwijderen of performance te verbeteren.

\section{Dagelijkse werkwijze}
Binnen Eaglesciende wordt er gewerkt middels het scrum principe. Er wordt gestracht om middels een full scrum manier te werken. Dit wil zeggen dat er teams volledig agile aan het werk zijn om iedere sprint van ongeveer 2 weken. een werkend product op te leveren middels een demo. Daarnaast wordt er gebruik gemaakt van een OTAP straat die gebruikt wordt in de verschillende fasen van ontwikkeling. Alle omgevingen worden door Jenkins op een Azure cloud gebuild. Om applicaties op te zetten wordt er gebruik gemaakt van een 'template' waarin veel boilerplate code staat die ervoor zorgt basis instellingen voor het bouwen van een project. Het voordeel is dat veel instellingen het zelfde zijn voor alle projecten en daar dus voor consistentie zorgen. In de template worden waarden aangepast benodigd voor het project.

Projecten worden in taken verdeeld en die worden individueel door een developer ontwikkeld en opgeleverd. het ontwikkelen gebeurt volledig lokaal op eigen machine waarna het getest wordt op een persoonlijke omgeving in Azure cloud. Als de taak voltooid is, wat wil zeggen dat er voldoende code coverage is middels unit tests en de taak is gereviewd door een collega. wordt deze terug gemerged in de een development-branch op git welke een build kan triggeren om een testomgeving te bouwen waar iedereen in het team op kan testen. Als alles volledig en juist is getest wordt er een snapshot genmaakt van deze branch en op een acceptatie platform gebuild, dit platform waarbij getracht wordt een zo correct mogelijke replica van productie te zijn. Op acceptatie kan de klant zelf onderzoeken of de software/ aanpassingen naar wens en requirements zijn. waarna er een deploy wordt gestart naar productie waar het leefd tot een volgende iteratie wordt gedeployed.

\section{Welke dev-stack wordt op het moment voornamlijk gebruikt binnen eagleScience?}

Binnen Eaglescience worden er in principe drie talen gebruikt voor het ontwikkelen van een applicatie binnen deze talen worden een aantal frameworks gebruikt. Naast de programeer talen maakt eagleScience gebruik van zowel SQL(MySQL) als NoSQL(MongoDB) databases om data in op te slaan.
\begin{itemize}
\item \textbf{Scala 2.xx} gekozen om functioneel programeren te ondersteunen maar ook de mogelijkheid om OOP methodieken te gebruiken. De filosofie binnen Eaglescience is dat functioneel programeren de testbaarheid en daarmee de zekerheid op goede software toeneemt als er puur funtioneel geprogrammeert wordt. Dit wil zeggen dat voor input x in een functie altijd output y is. Zonder enig side effect wat andere waarden muteerd. sterker nog er wordt binnen functioneel programeren bijna altijd gewerkt middels constanten die een waarde toegewezen krijgen. Mocht een veranderde waarde toch aangepast moeten worden wordt deze in een nieuwe constante geplaatst. De filosofie komt uit de wiskunde waarbij een input x nooit veranderd in een functie naar een andere waarde. Een ander voordeel van het gebruik maken van Scala is dat het draait in de Java Virtual Machine(JVM) en dus gebruik kan maken van Java bibliotheken. en Java Sourcecode. Binnen scala maken we gebruik van de volgende bibliotheken
\begin{itemize}
\item \textbf{PlayFramework 2.xx} Een web framework voor de ontwikkeling van webapplicaties in Scala we gebruikten het vooral als router voor de verschillende microservices die er achterliggen.
\item \textbf{archES} is een intern ontwikkeld framework wat de opbouw en de communicatie tussen microservices in scala verbeterd. archES is geinspireerd op Apache KAFKA en werkt middels de zelfde pub -> sub principe.
\iutem \textbf{GraphQL} Niet echt een Scala framework maar wel voor gebruikt door eaglescience als in de backend. GraphQL is een query taal voor API's en een runtime om deze queries te voorzien van bestaande data. GraphQL geeft de mogelijkheid om precies te vragen aan de backend wat er nodig is zonder een heel object en alle metadata op te halen.
\end{itemize}
\item \textbf{TypeScript}
\begin{itemize}
\item \textbf{Angular 12 > 14}
\item \textbf{ReactJS} 
\end{itemize}
\item \textbf{NativeScript}
\end{itemize}

\section{Wat wordt er gebruikt op het gebied van SOUP analyses?}
Op het moment van onderzoeken worden en er aantal tools gebruikt om inzichtelijk te maken welke kwetsbaarheden er



\section{Wat is Software of Unkown Pedigree(SOUP)?}
Volgens Wikipedia is \'software of unkown pedigree\' software dat niet volgens een software ontwikkel process of methode is ontwikkeld dat bekend is bij de eindgebruiker alsook software dat onbekende veiligheids eigenschappen heeft. De term wordt vooral gebruikt binnen het ontwikkelen van medische software.\\  % Bron: https://en.wikipedia.org/wiki/Software_of_unknown_pedigree
Door het gebruik van dit soort software kan men er dus nooit van uitgaan dat het veilig is.  SOUP komt veelvuldig voor in Open-source software, echter is closed software ook nooit gegarandeerd 100\% veilig als het van derden komt. Betekende dat eigen software niet altijd 100\% veilig is maar er is wel meer controle dan als er software van derden gebruikt wordt.

Deze definitie geld voor zowel volledige software pakketten als voor bibliotheken. Het gebruik van bibliotheken is iets waar Eaglescience veelvuldig gebruik van maakt. En geeft dit mogelijk problemen bij de veiligheid van de software zelf.

\section{Welk type bibliotheken kunnen onbekende veiligheid eigenschappen hebben of is de methode van ontwikkelen niet bekend?}
In principe is iedere bibliotheek waarvan niet de ontwikkelmethode of process te herleiden is, software of unkown pedigree. Veelal zijn dit de open-source bibliotheken die als onderdeel worden gebruikt in het ontwikkelen van zowel de frontend als de backend.

\subsection{OpenSource Software}Open-Source software staat er om bekend dat het veelal door een community wordt ontwikkeld waar bij de structuur niet altijd direct zichtbaar is en daarmee dus ook de methode niet altijd duidelijk.  	Eric S. Raymond spreek ook wel van een Bazaar model % Bron:https://en.wikipedia.org/wiki/Open-source_software#Development_model %bron : https://en.wikipedia.org/wiki/The_Cathedral_and_the_Bazaar
waarbij iedereen toegang heeft tot de source code en er zijn eigen aanpassingen aan doet, dit model stelt ook dat software snel gereleased wordt en er daarna frequente iteraties zijn om de software up-to-date te houden. Soms metd egebruiker als mede ontwikkelaar. Het grote voordeel is dat er meerdere inzichten van veel developers kunnen worden benut. dit geeft gelijk ook het nadeel dat het er vaak op neer komt dat er meerdere methodes gebruikt kunnen worden en daarmee dus niet kan worden vastgesteld hoe een pakket is ontwikkeld.

\subsection{Closed Source Software}% Bron:https://en.wikipedia.org/wiki/Proprietary_software
Closed source software is software dat is gebouwd onder een licentie, dit wil zeggen dat de gebruiker van de software/ bibliotheek meestal een bedrag moet betallen om het te betalen. Enkele voordelen van closed software is dat je weet wie het onwikkeld heeft en na een beetje onderzoek ook hoe de sotware gebouwd is. Al is dit laatste niet geheel relevant gezien er meestal een binary wordt geleverd zonder dat er aanpassingen vanuit de gebruiker in de source kan worden gedaan. Een nadeel is dat er meestal voor de software moet worden betaald en daarom dus duurder kan worden om het te gebuiken.

\subsection{Conclusie}

% tabel toevoegen https://en.wikipedia.org/wiki/Proprietary_software Types
het verschil tussen open en closed software zit hem vooral op de manier waarop het vervaardigd is, in een community of door een bedrijf. daarnaast zijn de kosten ook een verschil omdat er bij closes-source altijd een licentie moet worden verkregen en hier vaak (herhaaldelijke kosten) aan zitten. is dit bij open-source meestal
niet en wordt er vaak een vrijwillige bijdrage geleverd voor het gebruik ervan.

op basis van de kosten zou je kunnen zeggen dat open-source goedkoper is dan closed source bij de aanschaf. echter moet er rekening gehouden worden dat er niet altijd bekend is heo de software gebouwd is en wat de potenti\"ele gevaren hiervan zijn.

\section{Hoe wordt er op het dit moment een SOUP analyse uitgevoerd door Eaglescience en wat zijn de resource die gebruikt worden?}

Interview bouwen....


\lipsum[10]


\section{deelvraag 2}

\lipsum[10]

\section{deelvraag 3}

\lipsum[10]

\section{deelvraag 4}

\lipsum[10]
\section{deelvraag 5}

\lipsum[10]


\section{deelvraag 6}
\lipsum[10]
