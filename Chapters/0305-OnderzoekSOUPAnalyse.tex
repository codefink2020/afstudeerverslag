%TODO: MarginPars zetten
\chapter{Onderzoek: SOUP analyse}\label{ch:onderzoek:-soup-analyse} % Chapter title
Bronnen:

\begin{itemize}
    \item https://medium.com/@manjula.aw/nodejs-security-tools-de0d0c937ec0
    \item
\end{itemize}
\begin{itemize}
    \item Javascript     https://github.com/RetireJS/retire.js
    \item SBT https://github.com/albuch/sbt-dependency-check
    \item
\end{itemize}

Dit hoofdstuk geeft het onderzoek naar een methode om vanuit de projecten inzicht te krijgen in bekende kwetsbaarheden binnen gebruikte bibliotheken, en platformen. Er wordt ingegaan op welke tooling er beschikbaar is voor de verschillende onderdelen die we als EagleScience maken: Backend, frontend, Databases, Docker(containers). Daarna wordt er een selectie gemaakt voor de meest geschikte tool binnen de huidige buildstraat. Als laatst wordt er met de geselecteerde tooling een methode opgezet die het mogelijk maakt om gegevens uit de tooling te verkrijgen op een reproduceerbare manier.

\section{Onderzoeksvraag}\label{sec:onderzoeksvraag}
De hoofdvraag in dit onderzoek is: "Welke methodes zijn er beschikbaar voor het analyseren van kwetsbaarheden in bibliotheken van derden en welke het meest geschikt is voor het doel dat de opdrachtgever voor ogen heeft? ". Uit deze vraag komen een aantal deelvragen die ieders in een sectie hieronder wordt beantwoord.
\begin{itemize}
    \item Wat zijn de eisen waaraan de tool moet voldoen?
    \item Welke soorten tooling bestaan er?
    \item Welke tooling is er beschikbaar, wat zijn de voor en nadelen?
    \item Uit de gevonden tooling welke is het beste te integreren in de huidige pipeline/workflow?
\end{itemize}

\section{requirements}
De Opdracht luid dat er gescanned moet worden op actieve en niet actieve projecten op kwetsbaarheden. ER moet dus een methode komen die zowel periodiek een scan kan uitvoeren op projecten die niet meer actief in ontwikkeling zijn en een 'getriggerde' scan op het moment dat er bij een actief project code wordt toegevoegd en/of gewijzigd. Daarnaast moet de output van een suite te integreren zijn in de portal voor eenieder die hier belang bij heeft en de rechten bezit.

\subsection{Eisen aan de tool}
Om bruikbaar te zijn moet de tool aan een aantal eisen voldoen. De eisen die hier genoemd worden moet aan voldaan worden.
\begin{itemize}
    \item Tool moet output leveren op een reproduceerbare manier in een vaste vorm.
    \item Tool moet makkelijk te integreren zijn in de huide pipeline.
    \item Tool moet configureerbaar zijn, onderandere aangeven welke dependencies er overgeslagen moeten worden.
\end{itemize}
Deze eisen dienen te worden behaald anders kan de tool niet ingezet worden voor het doel dat voor ogen is.



\section{Beschikbare tools}
Er zijn veel tools beschikbaar die helpen met het veiliger maken van software. De tools zijn te verdelen in twee typen:

\subsection{Suites}
Suites zijn samengestelde pakketen vanuit meerdere tools die samen een volledige oplossing bieden op het gebied van software kwaliteit. Zo zijn er tools die checken of er herkenbare bugs zijn in een applicatie of dat er een bepaalde ingestelde coding wordt toegepast. Ook zijn er tools voor het zoeken naar kwetsbaarheden in geschreven code. De resulaten van deze tools worden samengevat in een rapport dat men inkan zien middels een web interface. Sommige suites geven ook de mogelijkheid om de evolutie van een project weer te geven zodat er een voortgang te zien in de ontwikkeling van veiligheid. Meeste van deze tools werken op basis van het maken van een snapshot tijdens een commit op een repository waardoor er gekeken wordt naar hoe de software gebouwd is op het moment van een oplevermoment.


\subsection{Commandline Tools}
Commandline tools worden in tegenstelling tot suites ontwikkeld voor een specifieke taak en doen deze taak volledig. zei genereren een output in textvorm welke dan al niet in een vast format(JSON, CSV) kan worden opgeslagen.
Commandline tools kunnen onderdeel zijn van suitesmaar veelal zijn dit open-source tools die de community heeft gemaakt om in een bestaande pipeline te integreren.



\subsection{voor en nadelen}
Beide soorten tools hebben voor en nadelen. Suites zorgen voor een algehele oplossing voor het beter maken van software maar focussen zich vooral op de kwaliteit van geschreven code, Met plugins kan er ook worden gekeken naar code toevoegingen vanuit bibliotheken van derden. Een nadeel is echter dat voor de ontwikkeltalen die EagleScience gebruikt er niet echt een defacto tool bestaat.

De voordelen van Commandline tools is dat deze makkelijker te integreren zijn in de huidige pipeline. Wat inhoud dat de werkwijze in principe niet veel hoeft te veranderen. Commandline tools worden ook vaak ontwikkeltaal specifiek uitgebracht. Wat in het geval van Scala een voordeel is gezien deze taal niet heel populair is vergeleken met Java, Java/TypeScript en C\#.


\subsection{Conclussie}
Uit de voor en nadelen die hierboven zijn genoemd kan worden geconcludeerd dat er een voorkeur is voor het gebruiken avn een Commandline Tool om de gewenste informatie te verkrijgen. Resulterende in een zoektocht naar een commandline tool die voor ieder component van de applicatie de gewenste resultaten kan geven. Zoals gemeld wordt er binnen EagleScience gebruik gemaakt van de volgende platformen / ontwikkeltalen.
\begin{itemize}
    \item \textbf{Development en productie platform} Docker containers in Kubernetes ( AKS van Azure)
    \item \textbf{Backend} geschreven in Scala
    \item \textbf{Portal} geschreven in Javascript gebruikmakend van dan al niet Angular CLI of React.
    \item \textbf{Database} MySQL of MongoDB in verschillende versie nagelang de requirements
    \item \textbf{App} geschreven in Nativescript.
\end{itemize}


\section{Welke tooling is geschikt voor het integreren met de huidige pipeline? }
Zoals hierboven gezegt is een Commandline tool handiger om te integreren in de huidige pipline.

Op de app na draaien alle componenten in een Docker container welke gehost worden op Azure. Er moet dus een analyse tool worden gevonden die we kunnen inzetten om dependencies voor Scala, JavaScript(NPM) en Docker images te kunnen analyseren.

Vulnerability scan is een 'hot' topic op het internet en een google search geeft al snel veel applicaties en tools die er voor kunnen zorgen dat er gescanned kan worden naar kwetsbaarheden in applicaties. Bij EagleScience zijn we op zoek naar een tool die we in projecten kunnen inzetten en er vervolgens zelf een applicatie omheen maken om intern de resultaten te kunnen gebruiken. We zijn dus opzoek naar plugin en of commandline tools die een output kunnen genereren in een bekend formaat zoals JSON of CSV.
