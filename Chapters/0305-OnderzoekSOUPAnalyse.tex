% Chapter X





\chapter{Onderzoek: SOUP analyse}\label{ch:onderzoek:-soup-analyse} % Chapter title
Bronnen:

\begin{itemize}
    \item https://medium.com/@manjula.aw/nodejs-security-tools-de0d0c937ec0
    \item
\end{itemize}
\begin{itemize}
    \item Javascript     https://github.com/RetireJS/retire.js
    \item SBT https://github.com/albuch/sbt-dependency-check
    \item
\end{itemize}

%TODO: MarginPars zetten
Dit hoofdstuk geeft het onderzoek naar een methode om vanuit de projecten inzicht te krijgen in bekende kwetsbaarheden binnen gebruikte bibliotheken, en platformen. Er wordt ingegaan op welke tooling er beschikbaar is voor de verschillende onderdelen die we als EagleScience maken: Backend, frontend, Databases, Docker(containers). Daarna wordt er een selectie gemaakt voor de meest geschikte tool binnen de huidige buildstraat. Als laatst wordt er met de geselecteerde tooling een methode opgezet die het mogelijk maakt om gegevens uit de tooling in de portal te krijgen.

\section{Onderzoeksvraag}\label{sec:onderzoeksvraag}
De hoofdvraag in dit onderzoek is:  "Welke methodes zijn er beschikbaar voor het analyseren van kwetsbaarheden in bibliotheken van derden en welke het meest geschikt is voor het doel dat de opdrachtgever voor ogen heeft? ". Uit deze vraag komen een aantal deelvragen die ieders in een sectie hieronder wordt beantwoord.
Deelvragen:
\smallskip
\begin{itemize}
    \item Wat zijn de eisen waaraan de tool moet voldoen?
    \item Welke soorten tooling bestaan er?
    \item Welke tooling is er beschikbaar, wat zijn de voor en nadelen?
    \item Uit de gevonden tooling welke is het beste te integreren in de huidige pipeline/workflow?

\end{itemize}

\section{requirements}
De Opdracht luid dat er gescanned moet worden op actieve en niet actieve projecten op kwetsbaarheden. ER moet dus een methode komen die zowel periodiek een scan kan uitvoeren op projecten die niet meer actief in ontwikkeling zijn en een 'getriggerde' scan op het moment dat er bij een actief project code wordt toegevoegd en/of gewijzigd. Daarnaast moet de output van een suite te integreren zijn in de portal voor een ieder die hier belang bij heeft en de rechten bezit.



\section{Zoeken naar tooling}
Er zijn veel tools beschikbaar die helpen met het veiliger maken van software. De tools zijn te verdelen in twee typen:
\begin{enumerate}
    \item Suites zijn volledige pakketen die de gebruiker is staat stelt om inzicht te krijgen in de huidige staat van de geschreven source code. Hier kan worden gedacht aan slechte geschreven code(codesmells), linting maar ook security issues. Suites maken gebruik van plugins om verschillende taken te kunnen uitvoeren en of te scannen. en dit maakt ze hoog configureerbaar op het moment dat er een plugin bestaat voor de te analyseren ontwikkeltaal/systeem.
    \item Command line tools die specifiek op een onderwerp scannen. De voodelen van deze manier is dat het makkelijk ergens in een bestaande oplossing is in te bouwen en dat deze tools over het algemeen goed configureerbaar zijn. Een nadeel kan zijn dat het tijd kan kosten om de gewenste output te verkrijgen en deze vervolgens 'mooi' weer te geven in een applicatie.
\end{enumerate}
Gezien er deze twee smaken zijn moet er gekeken worden welke van de twee het beste past is de workflow van EagleScience met in achtneming de requirements die opgelegd zijn door de opdrachtgever.

\subsection{suites}
Veel bedrijven houden zich bezig met het veiliger maken van software en er is een ware markt ontstaan in suites om hierbij te helpen. Bedrijven als Snyk, SonarSource, Veracode, Synopsis brengen suites uit die het mogelijk maken om applciaties te annalyseren met gebruik van zowel source als compiles code. Een naardeel is zoals eerder genoemd is dat veel van deze suites gespecialiseerd zijn in een enkele taal of middels plugins andere talen kunnen ondersteunen. Daarnaast zijn veel bedoelt om code quality omhoog te brengen om op die manier veiliger en betere software te maken. Het zoeken naar kwetsbaarheden in bibliotheken van derden is meestal een extra feature. Een ander nadeel is dat veel van deze pakketen onder een license vallen en er vaak veel voor betaald moet worden. En gezien het budget voor EagleScience niet heel groot is lijkt het voor de hand liggend om een andere weg te kiezen.
\subsection{Commandline tooling (CLI's)}
Commandline tooling is een manier van tooling dat zich zeer geschikt maakt om te integreren in een bestaande build pipeline. Veelal zijn CLI's ook configureerbaar in de mate hoever ze moeten scannen en welke output ze waar moeten geven. Omdat CLi's makkelijker te ontwikkelen zijn dan complete suites zijn er vaak voor iedere ontwikkeltaal wel een scanning tool beschikbaar.
\subsection{Conclussie}
Uit de voor en nadelen die hierboven zijn genoemd kan worden geconcludeerd dat er een voorkeur is voor het gebruiken avn een Commandline Tool om de gewenste informatie te verkrijgen. Resulterende in een zoektocht naar een commandline tool die voor ieder component van de applicatie de gewenste resultaten kan geven.





\begin{itemize}
    \item \textbf{Backend} geschreven in Scala
    \item \textbf{Portal} geschreven in Javascript gebruikmakend van dan al niet Angular CLI of React.
    \item \textbf{Database} MySQL of MongoDB in verschillende versie nagelang de requirements
    \item \textbf{App} geschreven in Nativescript.
\end{itemize}
Op de app na draaien alle componenten in een Docker container welke gehost worden op Azure. Er moet dus een analyse tool worden gevonden die we kunnen inzetten om dependencies voor Scala, JavaScript(NPM) en Docker images te kunnen analyseren.

Vulnerability scan is een 'hot' topic op het internet en een google search geeft al snel veel applicaties en tools die er voor kunnen zorgen dat er gescanned kan worden naar kwetsbaarheden in applicaties. Bij EagleScience zijn we op zoek naar een tool die we in projecten kunnen inzetten en er vervolgens zelf een applicatie omheen maken om intern de resultaten te kunnen gebruiken. We zijn dus opzoek naar plugin en of commandline tools die een output kunnen genereren in een bekend formaat zoals JSON of CSV.





\section{Security suites}
Op het moment van schrijven is de aanbod van security suites enorm veel bedrijven in de software security branch zien kansen om hier op in te spingen. Veel tools zetten in op het scannen van code welke geschreven worden door de teams zelf. Een enkele kan ook zoeken naar kwetsbaarheden die gemeld zijn als CVE echter zijn dit veelal plugins.
Een aantal voorbeelden van deze applicaties zijn:
\begin{itemize}
    \item \textbf{SonarQube} een suite dat veel analyses kan doen op code. Zoals codesmells, bugs, en security issues. Echter heeft deze suite welke wel geschikt is voor Java geen volledige ondersteuning voor Scala. Daarnaast komt er weer een applicatie bij die beheert moet worden.
    \item \textbf{}
\end{itemize}

\section{Backend scanning op kwetsbaarheden}
Binnen Eaglescience worden backend gebouwd in de taal Scala en op basis van een microservice architectuur. Daarnaast worden de dependencies die gebruikt worden in een file genaamd dependencies.scala. Er moet dus een tool komen die voor

Zoals gezegt wordt de backend voor applicaties geschreven in Scala. Daarnaast zijn de meeste backend binnen Eaglescience opgebouwd als multi-project Microservices Scala maakt gebruik van een dependencies.scala file om dependencies in aan te geven die in het gehele project ge
