% Chapter 3

\chapter{Onderzoeksmethode} % Chapter title

\label{OnderzoeksMethode} % For referencing the chapter elsewhere, use \autoref{ch:InOnderzoek}

De opdracht van de CTO luid: "Implementeer een oplossing om automatisch, periodiek een SOUP analyse te doen op project die zowel in productie draaien als in ontwikkeling zijn en geef hier een verslag van die in de portal te bekijken is." Om deze opdracht tot een goed einde te brengen wordt er eerst onderzoek gedaan naar de betekenis van een SOUP analyse en welke impact dit heeft op het bedrijf. Daarna zal er worden gekeken of er een dergelijk product op de markt is die zonder problemen in de huidige pipeline kan worden ge\"implementeert. Mocht dit niet mogelijk zijn zal er worden uitgeweken naar het implementeren van eigen oplossing. 


\section{Intake gesprek met opdrachtgever}
Om meer inzicht te krijgen in de details van de opdracht is er een intake gesprek gehouden. Dit intake gesprek is 1 op 1 gevoerd met de CTO waarna er een hoofdvraag is opgesteld 

Een verslag van het intakegesprek is te vinden in bijlage X%nog toewijzen.

\section{Onderzoeks vragen}
De opdracht vanuit Eaglescience geeft een aantal 


In dit deel wordt het onderzoek beschreven zoals deze is gedaan om inzicht te krijgen wat een SOUP analyse binnen Eaglescience in zou houden en welke oplossing er beschikbaar zou kunnen zijn. Mocht er geen oplossing zijn dan wordt overgegaan naar het ontwikkelen van een oplossing binnen Eaglescience zelf.

Aan de orde zullen komen:
\begin{itemize}
\item Interview met opdrachtgever om meer duidelijkheid en verdieping te verkrijgen over de opdracht.
\item Literatuur onderzoek naar het begrip SOUP en bijbehorende begrippen.
\item Marktonderzoek naar mogelijk beschikbare oplossingen van derden die binnen de eisen van de opdracht vallen.
\item Designonderzoek mocht er geen beschikbare oplossingen zijn.
\end{itemize}

Het resultaat van het onderzoek is dat er een oplossing wordt aangedragen die volgens de geldende eisen ge\"implementeert kan worden. Deze oplossing kan van derden zijn dan al niet zelfbouw. 






