% Chapter 3

\chapter{Onderzoeksmethode} % Chapter title

\label{OnderzoeksMethode} % For referencing the chapter elsewhere, use \autoref{ch:InOnderzoek}

De opdracht van de CTO luid: "Implementeer een oplossing om automatisch, periodiek een SOUP analyse te doen op project die zowel in productie draaien als in ontwikkeling zijn en geef hier een verslag van die in de portal te bekijken is." Om deze opdracht tot een goed einde te brengen wordt er eerst onderzoek gedaan naar de betekenis van een SOUP analyse en op welke manier het op dit moment wordt uitgevoerd. Daarnaast zal er gegeken worden naar de impact van de nieuwe methode op het bedrijf. Daarna zal er worden gekeken of er een dergelijk product op de markt is die zonder problemen in de huidige pipeline kan worden ge\"implementeert. Mocht dit niet mogelijk zijn zal er worden uitgeweken naar het implementeren van eigen oplossing.

\section{Intake gesprek met opdrachtgever}
Om meer inzicht te krijgen in de details van de opdracht is er een intake gesprek gehouden. Dit intake gesprek is 1 op 1 gevoerd met de CTO waarna er een hoofdvraag is opgesteld voor het onderzoek welke de basis was voor een aantal deelvragen die in het onderzoeksdeel wordt beantwoord. 

Een verslag van het intakegesprek is te vinden in bijlage X%nog toewijzen.

\section{Onderzoeks vragen}
De hoofdvraag voor het onderzoek luid als volgt: "Hoe kan Eaglescience een methode implementeren in de huidige pipeline zodat er periodiek een SOUP analyse gedaan wordt aan de software die in productie draait of op software waar op dit moment aan ontwikkelt wordt zonder dat dit onderdoet aan kwaliteit van de software en het gebruik van de huidige pipeline". Uit deze hoofdvraag komen een aantal deelvragen:
\begin{itemize}
\item Wat is SOUP en hoe analyseer ik dit?
\item Welke bibliotheken worden er gebruikt en in welke categorie vallen deze?
\item Hoe wordt er op het dit moment een SOUP analyse uitgevoerd door Eaglescience en wat zijn de resource die gebruikt worden?
\item Welke methoden zijn er buiten Eaglescience om te zien of een bibliotheek kwetsbaarheden bevat?
\item Wie zijn de uiteindelijke (eind)gebruikers van de module?

\item Welke pakketten zijn er te vinden die mogelijk binnen de eisen valt en passen in de pipeline van Eaglescience?
\item Waaruit bestaat de huidige pipeline?
\item Hoe gaat Eaglescience te werk, wat is het process dat gevolgd wordt?
\item Hoe relateert de pipeline zich tot het process binnen Eaglescience.
\end{itemize}

opsomming van de hoofd en deelvragen, er wordt nog geen antwoord gegeven!!!!!
\section{Onderzoekstypen}
\section{Onderzoeksmodel}
test
