
\chapter{Planning}\label{ch:planning} % Chapter title

\label{planning} % For referencing the chapter elsewhere, use \autoref{ch:InOnderzoek}

\section{Planning methode}\label{sec:planning-methode}
Binnen Eaglescience wordt er gewerkt middels de Agile Scrum methode, wat inhoud dat elk project incrementeel opgeleverd wordt in sprints.
De Gant Chart hieronder is dan ook ingedeeld in sprints van twee weken en geeft alleen de hoofd werkzaamheden weer.
De gedaileerde planning zal worden gedaan middels een Scrum board in Jira.
Welke iedere sprint zal worden gereviewed en aangepast aan de daadwerkelijke stand in het project.

\section{Project plannin in grote lijnen.}\label{sec:project-plannin-in-grote-lijnen.}
% aan werken om de charts gelijk te houden.... Wellicht gedraaid op de volgende pagina.
\begin{figure}
\begin{ganttchart}[hgrid=true,
vgrid={*2{red}, *1{green}, *{10}{blue, dashed}}, x unit=1.2cm, y unit title=.6cm, y unit chart=.6cm]{1}{11}
  \gantttitle{2021 sprints}{11}\\
  \gantttitlelist{1,...,11}{1} \\
  \ganttgroup{Ontwerp}{1}{4}\\
  \ganttbar{interviews}{1}{2}\\
  \ganttlinkedbar{uitwerking interviews}{2}{3}\\
  \ganttlinkedbar{Design plan}{3}{4}\\
  \ganttmilestone{design approved}{4}\\

  \ganttgroup{Architectuur}{4}{5}\\
  \ganttbar{onderzoek }{4}{4}\\
  \ganttlinkedbar{ module integratie}{5}{5}\\
  \ganttmilestone{Architectuur approved}{5}\\
  \ganttgroup{Implementatie}{6}{11}\\
  \ganttmilestone{implementatie done}{11}\\

\end{ganttchart}

\begin{ganttchart}[hgrid=true,
vgrid={*2{red}, *1{green}, *{10}{blue, dashed}}, x unit=2.4cm, y unit title=.6cm, y unit chart=.6cm]{1}{4}
  \gantttitle{2022 sprints}{4}\\
  \gantttitlelist{1,...,4}{1} \\
  \ganttgroup{Testing}{1}{4}\\
  \ganttmilestone{testing done approved}{4}\\

  \ganttgroup{Deployment}{2}{4}\\
  \ganttmilestone{Deployed}{4}\\
\end{ganttchart}
\caption{Planning}\label{fig:Planning}
\end{figure}

test
