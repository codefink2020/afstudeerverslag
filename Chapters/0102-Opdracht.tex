% Chapter 2

\chapter{Opdracht}\label{ch:opdracht} % Chapter title


 % For referencing the chapter elsewhere, use \autoref{ch:examples}
Tegenwoordig zijn software-bibliotheken niet meer weg te denken in het huidige software\-ontwikkelproces. Bibliotheken geven ontwikkelaars de mogelijkheid code te hergebruiken in meerdere projecten, om zo efficiënter te kunnen ontwikkelen. Dit helpt op zijn beurt om de Time-To-Market te verkorten. Bibliotheken kunnen door bedrijven zelf geschreven worden, in het geval van EagseScience is dit ArchES, of worden overgenomen van andere bedrijven/instellingen. ArchES is echter ook afhankelijk van een aantal bibliotheken die niet ontwikkeld zijn door Eaglescience. Hierdoor kan niet worden voorkomen dat bibliotheken worden gebruikt waarvan de afkomst niet geheel kan worden herleiden.

Deze (deels) onherleidbare bibliotheken vallen onder de noemer "Software of Unknown Provenance / Pedigree(SOUP)". Door het gebruik van dit soort bibliotheken kan er een aannemelijk risico ontstaan op het gebied van kwetsbaarheden. Om inzicht te krijgen in deze kwetsbaarheden en daarmee mogelijke veiligheidsissues dient er een SOUP-analyse gedaan te worden. Binnen EagleScience wordt het belang hiervan onderstreept en daarom wordt er gezocht naar een efficiënte en een waar mogelijk geautomatiseerde manier voor het uitvoeren van een dergelijke analyse om zo de veiligheid van de ontwikkelde applicaties te waarborgen zonder afbreuk te doen aan kwaliteit.

\section{Opdracht vanuit EagleScience}\label{sec:opdracht-vanuit-EagleScience}
Vanuit de CTO is de wens ontstaan om een systematisch opgebouwde methode te ontwikkelen waarbij er automatisch periodiek een SOUP-analyse gedaan wordt op bestaande en nieuwe projecten. Het beoogde resultaat is een module welke wordt toegevoegd aan de al bestaande portal van EagleScience waarbij project verantwoordelijken inzicht kunnen verkrijgen in de kwetsbaarheden die in een project aanwezig kunnen zijn door het gebruik van externe bibliotheken.

\subsection{Eisen aan de opdracht}\label{subsec: eisen-aan-de-opdracht}
Vanuit EagleScience zijn er een aantal eisen gesteld waaraan het eindproduct moet voldoen. Als er aan deze eisen is voldaan is er voor EagleScience een waardevol product welke gebruik kan worden genomen. Daarnaast zijn er een aantal opleveringseisen die gehaald dienen te worden om de kwaliteit te waarborgen.

\textbf{Functionele eisen}
\begin{itemize}
\item De module dient eenvoudig te kunnen worden gebruikt in de huidige Continuous Integration /Continuous Deployment (CI/CD) pipeline voor bestaande en nieuwe projecten
\item De module dient gebruik te maken van de bestaande huidige projectstructuur van het portal
\item De module dient ondersteuning te bieden voor meerdere omgevingen (OTAP)
\item De module wordt ontwikkeld in Angular en Play (scala), zodat het in het bestaande portal module past.
\item De module dient met een instelbaar interval de analyse uit te voeren
\item De module moet op project en omgevings niveau te rapporteren over bekende kwetsbaarheden
\item De module dient kwetsbaarheden op minimaal drie niveau’s in te schalen (kritisch, gemiddeld en laag)
\item De module dient ondersteuning te bieden voor het instellen van quality gates ten aanzien van de melding die het vind van ieder niveau, per project, per omgeving.
\end{itemize}
\textbf{Kwaliteitseisen}
\begin{itemize}
\item De module dient te voldoen aan de geldende kwaliteitsnormen binnen EagleScience, minimaal meetbaar door:
	\begin{itemize}
	\item test coverage > 70\%
	\item onderdeel van de bestaande CI/CD voor het EagleScience Portal
	\end{itemize}
\item De geschreven code dient gereviewd te worden door een EagleScience ontwikkelaar
\item In de module dient gescheiden componenten te bevatten: Frontend, Backend, API
\item Voor de API dient gebruik gemaakt te worden van swagger.
\item De module dient goed gedocumenteerd te zijn middels 'incode comments'.
\end{itemize}

\subsection{Deliverables vereiste resultaten}\label{subsec:deliverables-vereiste-resultaten}
Vanuit de CTO zijn er naast de functionele eisen ook eisen gesteld aan de oplevering:
\begin{itemize}
\item Geïntregreerde en aantoonbaar werkende module
\item De code van de module is gepubliceerd in Eaglsescience GitLab
\item Een handleiding over hoe de module gebruikt dient te worden
\item Eventuele aanvullende deliverables vanuit de HvA
\end{itemize}

\section{Opdracht fasen}\label{sec:opdracht-fasen}
[NOTE:] Eerst onderzoek afmaken dan aanpassen
Om de hierboven beschreven opdracht zo goed als mogelijk uit te voeren dient er eerst een onderzoek gedaan te worden naar begrippen binnen het domein SOUP, de ontwikkelomgeving van EagleScience en daarnaast naar mogelijkheden om bibliotheken te screenenen te testen op kwetsbaarheden. Na de onderzoeksfase moet er een module ontwikkeld worden die deze mogelijkheid implementeerd met inachtneming van de  hier boven genoemde eisen.

\subsection{Fase 1: Onderzoek} \label{subsec:fase-1:-onderzoek}
Als eeste dient er een onderzoek gedaan te worden naar de huidige situatie binnen EagleScience, waarbij er gekeken wordt naar de huidige dev-stack, de tooling, de werkwijze, als ook de huidige manier van uitrollen van applicaties. Daarna dient er een begrippen / literatuur onderzoek gedaan te worden binnen het domein SOUP om een goede kennis te vergaren over het domein om een basis te kunnen leggen voor een te implementeren module. Daarnaast dient er onderzoek gedaan te worden om te zien of er bibliotheken en resources zijn waar informatie over SOUP-bibliotheken te vinden is, en aan welke eisen deze bivbliotheken moeten voldoen om kwetsbaar te worden. Hier lettende op de eisen vanuit EagleScience en de mogelijkheden die deze analyse bibliotheken bieden. Deze fase wordt beschreven in het deel~\ref{prt:Onderzoek} van dit document.

\subsection{Fase 2: Oplevering SOUP analyse module}\label{subsec:fase-2:-oplevering-soup-analyse-module}
[NOTE:] verder uitwerken na onderzoek...
De uit het onderzoek behaalde resultaten ten aangaande beschikbare resources om een SOUP-analyse uit te voeren is een leidraad voor de implementatie van de module Deze module moet voldoen aan de eisen die gesteld zijn. Het ontwerp en implementatie wordt beschreven in deel~\ref{prt:Implementatie}
