
\chapter{Requirements Analyse}\label{ch:requirements-analyse}
Dit hoofdstuk beschrijft het probleem binnen het domein van EagleScience. Daarnaast wordt de huidige situatie beschreven waarna er een gewenste situatie wordt geschetst. De stakeholders worden defineert en in kaart gebracht Waarna hun requirements in kaart worden gebracht en middels de MoSCoW methode op prioriteit worden gezet. Het resultaat is een Requirements Specificatie die in zijn geheel is terug te vinden in Appendix~\ref{ch:requirements-specificatie}

\section{Huidige situatie}\label{sec:huidige-situatie}
EagleScience is al een geruime tijd bezig met het fine-tunen van de buildstraat en zaken te automatiseren om zo steeds efficienter software uit te kunnen rollen. Daarnaast is het bouwen van veilige software één van de hoofdpunten van aandacht. Om dit te kunnen garanderen is iedere ontwikkelaar verplicht om te onderzoeken welke gevolgen het gebruik van bepaalde bibliotheken heeft op de ontwikkelde software. Dit onderzoek wordt op dit moment voor een groot deel handmatig gedaan door het uitpluizen van dependency declaraties op verschillende sites. Dit proces is tijdrovend en de resulterende documentatie is vaak alleen op project niveau beschikbaar. Daarnaast zijn er al afgeronde applicaties die niet onder directe aandacht van het team staan, maar wel worden gehost door EagleScience. Met de tijd kunnen er kwetsbaarheben ontstaan die ongemerkt blijven bestaan. En er mogelijk pas actie wordt ondernomen als er middels de applicatie een aanval of inbraak is gedaan.

\section{Gewenste situatie}\label{sec:gewenste-situatie}
Een situatie waar EagleScience naar toe wil is dat er periodiek of middels een trigger \footnote{Er kan bijvoorbeeld gedacht worden aan een commit op een branch zodat niet iedere dag een update wordt gedaan op De acceptence branche is een goed voorbeeld hiervoor} wordt onderzocht of er in de huidige dependency tree bibliotheken zitten die mogelijk kwetsbaarheden bevatten. Deze kennis dient gedeelt te worden door middel van een module in de portal die al reeds gebruikt wordt door EagleScience. Door de resultaten weer te geven in de portal ontstaat er een beter inzicht in welke bibliotheken we gebruiken en welke er potentieel kwetsbaarheden bevatten. Wat op zijn beurt weer voor veiligere applicaties kan zorgen. Ook voor de applicaties waar niet meer actief op ontwikkeld wordt. \footnote{Het is natuurlijk wel zo dat er een afhankelijk onstaat van externe bronnen die bekend moeten maken dat er een kwetsbaarheid is.}



%Om inzicht te krijgen in de eisen van de nieuwe module naast de eisen die al vermeld staan in de opdracht is er een intake gesprek geweest met de CTO (Bas Breier), in dit gesprek is aan bod gekomen wie de betrokkenen zijn en welke requirements hij heeft naast de requirements die in de opdracht staan.
%In dit gesprek is er meer ingegaan op de details van het functioneren van de module.
%Ook is er een beeld geschets over de huidige situatie en naar welke situatie er gegaan moet worden. %TODO: nog even aanpassen in ABN Het verslag is
%Zie in Appendix~\ref{sec:intake-gesprek} het verslag van dit interview.
%\bigskip
%\section{Huidige situatie}\label{sec:huidige-situatie2}
%In de huidige situatie wordt er een SOUP-analyse gedaan door de ontwikkelaars op het moment dat een project ontwikkeld wordt.
%Dit is vaak handmatig zoeken in online resources op bibliotheken die gebruikt worden.
%En deze worden vervolgen geupdate waar nodig.
%Dit zoeken naar resources is tijdsintensief.
%En daarnaast is de rapportage ook niet centraal terug te vinden.
%Dit neemt veel kostbare tijd in beslag die beter besteed kan worden om nieuwe features toe te voegen.
%Daarnaast worden de bevindingen die gedaan worden niet centraal opgeslagen zodat er een potentie is dat niet iedereen op de hoogte is van de actuele informatie.
%
%\section{Gewenste situatie}\label{sec:gewenste-situatie2}
%De gewenste situatie is dat projectmanagers, ontwikkelaars en het dagelijks bestuur real-time inzage hebben in de huidige staat van de kwetsbaarheden in de gebruikte externe bibliotheken.
%Dit is te doen door een onderdeel in een portal te bouwen die op een overzichtelijke manier deze informatie weergeeft.
%
%Om de informatie weer te kunnen geven moet er een manier worden gevonden om tijdens een bouwprocess de versies van de gebruikte bibliotheken te achterhalen.
%En deze vervolgens tegen een Vulnerability Database te leggen.
%Deze gegevens dienen in een interne database opgeslagen te worden waarop de het onderdeel in de portal gegevens op kan halen.
%Waarop vervolgens door de stakeholders de gewenste informatie gehaald kan worden.

\section{De stakeholders}\label{sec:de-stakeholders}
Binnen EagleScience zijn er een aatal stakeholders die belang hebben bij deze nieuwe methode en module.Naast de stakeholders binnen Eaglescience is er nog een stakeholder in de vorm van de klant. Hoewel deze niet actief deelneemt in de ontwikkeling van deze module heeft het er toch baat bij dat deze module tot stand komt met als resultaat dat zij veiligere software geleverd krijgen. Met personen uit de interne stakeholders groepen zijn interviews gehouden om een inzicht te krijgen in hun belang en invloed bij de module.
Daarnaast is er een eerste lijst met requirements opgesteld waaraan de module dient te voldoen.
Deze lijst is een start en zal na enkele sprintdemo's worden aangepast of uitgebreid.
Na iedere tweede sprint zal een evaluatie worden gehouden om inzicht te krijgen of de requirements nog accuraat zijn en eventueel nog moeten worden aangescherpt.
De belangen en invloeden worden in de komende subsecties verder toegelicht.

\subsection{Dagelijks bestuur (intern)}\label{subsec:dagelijks-bestuur-(intern)}
Het dagelijks bestuur ziet vooral voordelen in het inzicht krijgen van kwetsbaarheden op een overzichtelijke manier, zodat ze kunnen sturen in het gebruik van biblioteken of andere technologiën. Ook al zullen er kosten die niet direct terug te verdienen zijn gemoeit met de ontwikkeling van een nieuwe methode.
Echter, zien zij ook kosten gemoeid met de verandering.
Door de manier van werken dienen deze kosten terug verdient te worden door werkzaamheden binnen andere projecten.
De CTO ziet vooral tijdswinst zodat de time-to-market voor andere projecten hoger ligt en dus meer verdient kan worden.
\subsection{Projectmanagers (intern)}\label{subsec:projectmanagers-(intern)}
Project managers krijgen op dit moment een update over de staat van kwetsbaarheden tijdens stand-ups en aan het einde van een sprint tijdens de sprint demo's.
De nieuwe module biedt ze de mogelijkheid om up-to-date informatie on-demand te verkrijgen.
Op de vraag of het het waard is dat een aantal ontwikkelaars tijd kwijt zijn in testen en meedenken over de module weegt volgens hen op tegen de voordelen die de module in de toekomst kan brengen.
\subsection{Ontwikkelteam (intern)}\label{subsec:ontwikkelteam-(intern)}
Het ontwikkelteam wil graag meedenken en meewerken aan een oplossing, gezien zij de gene waren die handmatig de analyse uitvoerden.
Zij zien voor een oplossing voor een taak dat veel tijd in beslag nam en afleide van de daadwerkelijke taak.
%\subsection{Klant (extern)}\label{subsec:klant-(extern)}
%Als laatste de klant welke een passieve stakeholder is gezien zij niet direct betrokken zijn bij de ontwikkeling van de module maar wel verbeteringen genieten in de zin van veilige en betrouwbare software.

\subsection{Stakeholder analyse}\label{subsec:stakeholder-analyse}
\begin{figure}[H]
\myfloatalign
\includegraphics[width=10cm]{gfx/stakeholderanalyse}
\caption{StakeHolders Analyse}
\label{fig:StakeholderAnalyse}
\end{figure}
Zoals te zien is in figuur~\ref{fig:StakeholderAnalyse} zijn de projectmanager, het ontwikkelteam en de klanten het meest gebaad bij een nieuwe module voor de analyse van kwetsbaarheden.
Echter zijn de klanten niet tot bijna niet betrokken bij de ontwikkeling van de module maar hebben er indirect wel belang bij omdat de software die voor hen ontwikkeld wordt veiliger wordt door het voeren van een geautomatiseerde analyse.
Door deze analyse worden alleen de requirements meegnomen die intern zijn opgenomen.
\section{Requirements}\label{sec:requirements}
Naast het analyseren van de betrokkenheid en belang van de stakeholders is er ook gevraagd welke requirements ze terug wilden zien in de applicatie en welke prioriteit er aan gesteld werdt.
Om een leidraad te verschaffen is de MoSCoW-methode gebruikt.
Hieronder is een lijst geformuleert met de belangrijkset requirements vanuit de stakeholders.
Deze lijst is niet volledig en wordt na iedere sprint aangepast aan de resultaten van de sprint ervoor.

\textbf{Must Have Moet nog onderverdeeld worden in MoSCoW}
\begin{itemize}
  \item Als \textit{gebruiker} wil ik dat de SOUP module in de portal te vinden is zodat alle tools die gebruikt worden binnen Eaglescience op een enkele plek te vinden zijn.
  \item Als \textit{gebruiker} wil ik een overzicht per project kunnen zien met daarin de gebruikte bibliotheken zodat ik inzage heb ik wat er gebruikt wordt voor ontwikkeling.
  \item Als \textit{gebruiker} wil ik een overzicht per project zien welke kwetsbaarheden er zich in bibliotheken bevinden, zodat ik actie kan ondernemen om de software nog veiliger te maken.
  \item Als \textit{gebruiker} wil ik in kunnen loggen met mijn LDAP? account zodat ik niet nog een keer een username/wachtwoord combinatie hoe te leren.
  \item Als \textit{gebruiker} wil ik een project kunnen toevoegen zodat ik ook van dat project de kwetsbaarheden in kan zien en deze software ook veilger wordt.
  \item Als \textit{Module} wil ik een update krijgen van de laatste build met specifiek de laatste kwetsbaarheden, zodat ik deze kan weergeven in de portal.
  \item Als \textit{module} wil ik
  \item Als \textit{gebruiker} wil ik dat periodiek automatisch een check analyse wordt uitgevoerd zodat ik er zelf niet naar om hoef te kijken.
  \item Als \textit{gebruiker} wil ik zelf een analyse kunnen starten voor een project zodat ik een up-to-date versie heb van de resultaten.
  \item Als \textit{Project manager} wil ik projecten kunnen toevoegen aan de module, zodat ook deze mee genomen worden in de automatische analyse.
  \item Als \textit{Project manager} wil ik ontwikkelaars kunnen toevoegen aan een project zodat deze ook inzicht krijgen in de huidige stand van zaken.
  \item Als \textit{Project manager} wil ik een notificatie( via mail/rocketchat) ontvangen als er een
\end{itemize}

\textbf{Should Have}
\begin{itemize}
  \item Moeten nog voorkomen uit de prioriteit analyse
\end{itemize}

\textbf{Could Have}
\begin{itemize}
\item
\end{itemize}

\textbf{Won't Have}
\begin{itemize}
  \item Moeten nog voorkomen uit de prioriteit analyse
\end{itemize}
De Won't haves staan hierbij genoemd als leidraad voor eventueel updates in de toekomst.
Als blijkt dat er tussen de won'ts toch low hanging fruit blijkt te hangen kunnen deze meegenomen worden in de sprints.
De requirements worden als epics in een JIRA omgeving gezet om vervolgens een planning te kunnen maken.



\section{kennis vergaring}
De secties over de huidige en gewenste situaties zijn voornamlijk gebasseerd op eigen observaties en gesprekken die ik heb gehad met enkele ontwikkelaars en de CTO. Uit deze gesprekken kwamen ook een aantal stakeholders naar voren die ook in het project moeten worden meegenomen. Deze zijn geinterviewd om informatie te verkrijgen over de wensen die zij hebben voor de nieuwe module maar ook het belang die zij hebben in de module. Het geen resulteerd in een document dat te lezen is in \ref{ch:requirements-specificatie}
